\appendix
\appendixpage
\addappheadtotoc 

\section{Raccolta algoritmi}

\subsection{Insertion Sort}
Per approfondire, vedi la sezione \ref{insertionsort} 
\begin{codebox}
\Procname{\proc{InsertionSort}$(A)$}
\li $n \gets \attrib{A}{length}$
\li \For $j \gets 2$ \To $n$
	\Comment il primo elemento è già ordinato
\li		\Do
			$\id{key} \gets A[j]$ 
			\Comment $A[1 \twodots j-1]$ ordinato
\li			$i \gets j-1$
\li			\While $i > 0$ and $A[i] > \id{key}$
\li				\Do
					$A[i+1] \gets A[i]$
\li					$i \gets i-1$
				\End
\li			$A[i+1] \gets \id{key}$
		\End
\end{codebox}

\subsection{Merge Sort}
Vedi la sezione \ref{mergesort}
\begin{codebox}
\Procname{\proc{MergeSort}$(A,p,r)$}
\li \If $p < r$
\li     \Then
            $q \gets \floor{\frac{p + r}{2}}$ 
        \Comment arrotondato per difetto
\li         $\proc{MergeSort}(A,p,q)$
        \Comment ordina \texttt{A[p$\twodots$q]}
\li         $\proc{MergeSort}(A,q+1,r)$
        \Comment ordina \texttt{A[q+1$\twodots$r]}
\li         $\proc{Merge}(A,p,q,r)$
        \Comment ``Merge'' dei due sotto-array 
        \End
\end{codebox}
\begin{codebox}
\Procname{\proc{Merge}$(A,p,q,r)$}
\li $\id{n1} \gets q-p+1$
    \Comment gli indici partono da 1
\li $\id{n2} \gets r-q$
\zi \Comment \kw{L} sotto-array sx, \kw{R} sotto-array dx
\li \For $i \gets 1$ \To $\id{n1}$
\li     \Do
            $L[i] \gets A[p+i-1]$
        \End
\li	\For $j \gets 1$ \To $\id{n2}$
\li		\Do
			$R[j] \gets A[q+j]$
        \End
\li         $L[\id{n1}+1] \gets R[\id{n2}+1] \gets \infty$
\li         $i \gets j \gets 1$
\li         \For $k = p$ \To $r$
\li             \Do
                    \If $L[i] \leq R[j]$
\li                     \Then
                            $A[k] \gets L[i]$
\li                         $i \gets i+1$
\li                     \Else
                    \Comment \texttt{L[i]} $>$ \texttt{R[j]}
\li                            $A[k] \gets R[j]$
\li                         $j \gets j+1$
                        \End
                \End
\end{codebox}

\subsection{Insertion Sort ricorsivo}

\begin{codebox}
\Procname{\proc{InsertionSort}$(A,j)$}
\li \If $j > 1$
\li     \Then
		$\proc{InsertionSort}(A,j-1)$
        \Comment ordina \texttt{A[1$\twodots$j-1]}
\li         $\proc{Insert}(A,j)$
        \Comment inserisce \texttt{A[j]} in modo ordinato in \texttt{A}
        \End
\end{codebox}
\begin{codebox}
\Procname{$\proc{Insert}(A,j)$}
\zi	\Comment Precondizione: \texttt{A[1$\twodots$j-1]} è ordinato
\li	\If $(j > 1) \ \kw{and} \ (A[j] < A[j-1])$
\li		\Then
			$A[j] \leftrightarrow A[j-1]$
		\Comment scambia le celle \texttt{j} e \texttt{j-1}
\zi		\Comment se le celle sono state scambiate, ordina 
\zi		\Comment il nuovo sottoarray \texttt{A[1$\twodots$j-1]}
\li			$\proc{Insert}(A,j-1)$
		\End
\end{codebox}

\subsubsection{Correttezza di InsertionSort(A, j)}
Procediamo per induzione:
\begin{itemize}
	\item[] $(j \leq 1) \quad$ Caso base. Array già ordinato, non faccio nulla $\Rightarrow$ ok;
	\item[] $(j > 1) \quad$ Per ipotesi induttiva, la chiamata \texttt{Insertion-Sort(A,j-1)}
	ordina \texttt{A[1$\twodots$j-1]}. Assumendo la correttezza di \texttt{Insert(A,j-1)}, esso
	``inserisce'' \texttt{A[j]} $\Rightarrow$ produce \texttt{A[1$\twodots$j]} ordinato.
\end{itemize}

\subsubsection{Correttezza di Insert(A, j)}
Anche qui, dimostrazione per induzione:
\begin{itemize}
	\item[] $(j = 1) \quad$ Caso base. \texttt{A[1]} da inserire nell'array vuoto. Non fa nulla
	$\Rightarrow$ ok;
	\item[] $(j > 1) \quad$ Due sottocasi:
	\begin{itemize}
		\item \texttt{A[j]} $\geq$ \texttt{A[j-1]}: non faccio nulla, \texttt{A[1$\twodots$j]} già
		ordinato;
		\item \texttt{A[j]} $<$ \texttt{A[j-1]}: scambio le chiavi delle due celle. Il nuovo \texttt{A[j]}
		sarà sicuramente maggiore di qualsiasi altro elemento che lo precede, poiché, per precondizione di 
		\texttt{Insert}, \texttt{A[1$\twodots$j-1]} era ordinato, e dato che valeva \texttt{A[j-1]} $\geq$ 
		\texttt{A[j]}, il nuovo \texttt{A[j]} (che è il precedente \texttt{A[j-1]}) sarà sicuramente l'elemento con il valore più alto. 
		Dopodichè, chiamo \texttt{Insert(A,j-1)} per ordinare la cella \texttt{A[j-1]}.
 	\end{itemize}
\end{itemize}
 
\subsection{CheckDup}
Algoritmo che verifica la presenza di duplicati in \texttt{A[p$\twodots$r]} e, 
solo se non ci sono, ordina l'array.

Se \texttt{A[p$\twodots$q]} e \texttt{A[q+1$\twodots$r]} ordinati e privi di duplicati:
\begin{itemize}[noitemsep]
	\item Se \texttt{A[p$\twodots$r]} non contiene duplicati, ordina e restituisce \texttt{false};
	\item altrimenti, restituisce \texttt{true}.
\end{itemize}

\begin{codebox}
\Procname{\proc{Check-Dup}$(A,p,r)$}
\li \If $p < r$
\li     \Then
            $q \gets \floor{\frac{p + r}{2}}$ 
        \Comment arrotondato per difetto
\li         \Return $\proc{Check-Dup}(A,p,q)$
\li         \mbox{ or } $\proc{Check-Dup}(A,q+1,r)$
\li         \mbox{ or } $\proc{DMerge}(A,p,q,r)$
        \End
\end{codebox}
\begin{codebox}
    \Procname{$\proc{DMerge}(A,p,q,r)$}
    \li $\id{n1} \gets q-p+1$
        \Comment gli indici partono da 1
    \li $\id{n2} \gets r-q$
    \zi \Comment \kw{L} sotto-array sx, \kw{R} sotto-array dx
    \li \For $i \gets 1$ \To $\id{n1}$
    \li     \Do
                $L[i] \gets A[p+i-1]$
            \End
    \li	\For $j \gets 1$ \To $\id{n2}$
    \li		\Do
                $R[j] \gets A[q+j]$
            \End
    \li         $L[\id{n1}+1] \gets R[\id{n2}+1] \gets \infty$
    \li         $i \gets j \gets 1$
    \li         \While $(k \leq p) \kw{ and } (L[i] \neq R[j])$ 
    \li             \Do
                        \If $L[i] < R[j]$
    \li                     \Then
                                $A[k] \gets L[i]$
    \li                         $i \gets i+1$
    \li                     \Else
                        \Comment \texttt{L[i]} $>$ \texttt{R[j]}
    \li                            $A[k] \gets R[j]$
    \li                         $j \gets j+1$
                            \End
    \li                 $k \gets k+1$
                    \End
    \li         \Return $k \leq r$
\end{codebox}


\subsubsection{Correttezza di DMerge(A,p,q,r)}
\begin{itemize}
	\item \texttt{A[p$\twodots$k-1]} è ordinato, contiene \texttt{L[1$\twodots$i-1]}$\cup$\texttt{R[1$\twodots$j-1]};
	\item \texttt{A[p$\twodots$k-1]} $<$ \texttt{L[1$\twodots$n1]}, \texttt{R[1$\twodots$n2]}.
\end{itemize}

\subsection{SumKey}
Dato \texttt{A[i$\twodots$n]} e \texttt{key} intera, \texttt{Sum(A,key)} restituisce:
\begin{itemize}
	\item \texttt{true} se $\exists i, j \in [1,n] : key = A[i] + A[j]$;
	\item \texttt{false} altrimenti.
\end{itemize}

Vediamo una prima versione, non efficiente, dell'algoritmo. Ha complessità $O(n^2)$.
\begin{codebox}
\Procname{\proc{SumB}$(A,\id{key})$}
\li	$n \gets \attrib{A}{length}$
\li	$i \gets j \gets 1$
\li \While $(i \leq n) \kw{ and } (A[i] + A[j] \neq \id{key})$
\li 	\Do
			\If $j = n$
\li				\Then
					$i \gets i + 1$
\li				\Else
\li					$j \gets j + 1$
				\End
		\End
\li	\Return $i \leq n$				
\end{codebox}

Ecco ora una versione più efficiente, che però richiede un \texttt{sorting} preventivo, che quindi causa \emph{side effect}. Si assume un 
algoritmo di sorting con complessità $O(n \log n)$. Con questa premessa, la ricerca della coppia di valori
ha complessità $O(n)$ nel caso peggiore. Nel complesso, vale quindi:
$$O(n \log n + n) = O(n \log n)$$
\begin{codebox}
\Procname{\proc{Sum}$(A,\id{key})$}
\li	$n \gets \attrib{A}{length}$
\li	\proc{sort}$(A)$
		\Comment complessità $O(n \log n)$
\li	$i \gets 1, \ j \gets n$
\li \While $(i \leq j) \kw{ and } (A[i] + A[j] \neq \id{key})$
\li 	\Do
			\If $A[i] + A[j] < \id{key}$
\li				\Then
					$i \gets i + 1$
\li				\Else
\li					$j \gets j - 1$
				\End
		\End
\li	\Return $i \leq j$	
\end{codebox}

\subsubsection{Correttezza di Sum(A, key)}
Valgono i seguenti invarianti:
\begin{enumerate}[label=(\arabic*)]
	\item \label{sum:cond1}$\forall h \in [1, i-1], \; \forall k \in [h, n] \Rightarrow A[h] + A[k] \neq key$
	\item \label{sum:cond2}$\forall k \in [j+1,n], \; \forall h \in [1,k] \Rightarrow A[k] + A[h] \neq key$
\end{enumerate}
Supponiamo di trovarci in $A[i] + A[j] < key$
\begin{itemize}[noitemsep]
	\item[$\rightarrow$] incremento $i$;
	\item[\ref{sum:cond1}] \textbf{non} cambia;
	\item[\ref{sum:cond2}] (vogliamo dimostrare) $\forall k \in [i,n] \quad A[i] + A[k] \neq key$.\par
		Distinguiamo \underline{2 casi}.
		\begin{itemize}
			\item Siccome vale $A[k] \leq A[j]$, allora
				$$A[i] + A[k] \leq A[i] + A[j] > key$$
			\item $k \in [j+1,n]$ quindi
			$$A[i] + A[k] \neq key \text{ per (2)}$$
		\end{itemize}
		Se esco perché $i > j$, \textbf{non} c'è una soluzione poiché
		$$(1) + (2) \Rightarrow \forall h \leq k \quad A[h] + A[k] \neq key$$ 
\end{itemize}

Presetiamo ora una terza soluzione, che però richiede un costo in memoria direttamente proporzionale al valore $max$ (che chiameremo $top$)
dell'array considerato, poiché richiede di allocare un array $V$ di booleani di dimensione dipendente da $top$, in cui il valore \texttt{A[i]} corrisponde alla cella \texttt{V[A[i]]}. Assumiamo
\begin{gather*}
	A[i] \geq 0 \quad \forall i \in [i,n], \ key \leq top \\
	V[v] = \text{\texttt{true} sse } \exists i : A[i] = v 
\end{gather*}
\begin{codebox}
\Procname{\proc{SumV}$(A,\id{key})$}
\li	$V[0 \twodots \id{key}] \leftarrow \const{false}$
		\Comment $\Theta (key) = O(top) = O(1)$
\li	$i \gets 1$
\li $\id{found} \gets \const{false}$
\li \While $(i \leq n) \kw{ and } not \ \id{found}$
\li 	\Do
			\If $A[i] \leq \id{key}$
\li				\Then
					$V[A[i]] \gets \const{true}$
\li					$\id{found} \gets V[\id{key} - A[i]]$
				\End
\li 		$i \gets i + 1$
		\End
\li	\Return $\id{found}$
\end{codebox}

Complessità:
\begin{itemize}
	\item $O(n)$ se $top$ costante;
	\item $O(n \cdot key)$ altrimenti.
\end{itemize}
\newpage
\subsection{Heap Sort}
Per approfondire, vedi \ref{heapsort}. 
\begin{codebox}
\Procname{\proc{Left}$(i)$}
\zi \Comment restituisce il figlio sx del nodo $i$
\li	\Return $2*i$
\end{codebox}
\begin{codebox}
\Procname{\proc{Right}$(i)$}
\zi \Comment restituisce il figlio dx del nodo $i$
\li	\Return $2*i+1$
\end{codebox}
\begin{codebox}
\Procname{\proc{Parent}$(i)$}
\zi \Comment restituisce il genitore del nodo $i$
\li	\Return $\floor{i/2}$
\end{codebox}
\begin{codebox}
\Procname{\proc{MaxHeapify}$(A,i)$}
\li $l \gets \proc{Left}(i)$
\li $r \gets \proc{Right}(i)$
\li \If $(l \leq \attrib{A}{heapsize}) \kw{ and } (A[l] > A[i])$
\li 	\Then
			$\id{max} \gets l$
\li		\Else 
\li			$\id{max} \gets i$			
		\End
\li \If $(r \leq \attrib{A}{heapsize}) \kw{ and } (A[r] > A[max])$
\li 	\Then
			$\id{max} \gets r$
		\End
\li \If $(\id{max} \neq i)$
\li 	\Then 
			$A[i] \leftrightarrow A[\id{max}]$
\li			$\proc{MaxHeapify}(A,\id{max})$
		\End
\end{codebox}
\begin{codebox}
\Procname{\proc{BuildMaxHeap}$(A)$}
\li $\attrib{A}{heapsize} = \attrib{A}{length}$
\li \For $i \gets \floor{\attrib{A}{length}/2} \kw{ down} \To 1$
\li 	\Do
			$\proc{MaxHeapify}(A,i)$
		\End
\end{codebox}
\begin{codebox}
\Procname{\proc{HeapSort}$(A)$}
\li $\proc{BuildMaxHeap}(A)$ 
    \Comment $O(n)$
\li \For $i \gets \attrib{A}{length}$ \kw{down} \To $2$
\li     \Do
            $A[1] \leftrightarrow A[i]$
\li         $\attrib{A}{heapsize} \gets \attrib{A}{heapsize} - 1 $
\li         $\proc{MaxHeapify}(A,1)$
                \Comment $O(\log n)$
        \End
\end{codebox}

\subsection{Code con Priorità}
(Sezione \ref{codeconpriorita})
\begin{codebox}
\Procname{\proc{Max}$(A)$}
\li \If $\attrib{A}{heapsize} = 0$
\li     \Then
            \kw{error}
\li     \Else
            \Return $A[1]$
        \End
\end{codebox}
\begin{codebox}
\Procname{\proc{ExtractMax}$(A)$}
\li $\id{max} \gets A[1]$
\li $A[1] \gets A[\attrib{A}{heapsize}]$
\li $\attrib{A}{heapsize} \gets \attrib{A}{heapsize} - 1$
\li $\proc{MaxHeapify}(A,1)$
        \Comment ripristina le proprietà di \emph{MaxHeap}
\li \Return $\id{max}$
\end{codebox}
\begin{codebox}
\Procname{\proc{MaxHeapifyUp}$(A,i)$}
\li \If $(i > 1) \kw{ and } (A[i] > A[\proc{Parent}(i)])$
\li     \Then
            $A[i] \leftrightarrow A[\proc{Parent}(i)]$
\li         $\proc{MaxHeapifyUp}(A,\proc{Parent}(i))$
        \End
\end{codebox}
\begin{codebox}
\Procname{\proc{Insert}$(A,x)$}
\li $\attrib{A}{heapsize} = \attrib{A}{heapsize} + 1$
\li $A[\attrib{A}{heapsize}] \gets x$
\li $\proc{MaxHeapifyUp}(A,\attrib{A}{heapsize})$
\end{codebox}
\begin{codebox}
\Procname{\proc{IncreaseKey}$(A,i,\delta)$}
\zi \Comment \emph{Precondizione}: $\delta \geq 0$
\li $A[i] \gets A[i] + \delta$
\li $\proc{MaxHeapifyUp}(A,i)$
\end{codebox}
\begin{codebox}
\Procname{\proc{ChangeKey}$(A,i,\delta)$}
\li $A[i] \gets A[i] + \delta$
\li \If $\delta > 0$
\li     \Then
            $\proc{MaxHeapifyUp}(A,i)$
\li     \Else
        \Comment $\delta \leq 0$
\li         $\proc{MaxHeapify}(A,i)$
        \End
\end{codebox}
\begin{codebox}
\Procname{\proc{DeleteKey}$(A,i)$}
\li $\id{old} \gets A[i]$
\li $A[i] \gets A[\attrib{A}{heapsize}]$
\li $\attrib{A}{heapsize} \gets \attrib{A}{heapsize} - 1$
\li \If $\id{old} \leq A[i]$
\li     \Then
            $\proc{MaxHeapifyUp}(A,i)$
\li     \Else
\li         $\proc{MaxHeapify}(A,i)$
        \End
\end{codebox}