\begin{codebox}
\Procname{$\proc{Left}(T,x)$}
\li $\attrib{x}{right} \gets \attrib{y}{left}$
\li $\attrib{\attrib{x}{right}}{p} \gets x$
\li $\attrib{y}{left} \gets x$
\li $\attrib{x}{p} \gets y$
\li $\proc{Transplant}(T,x,y)$
\li $\attrib{y}{size} \gets \attrib{x}{size}$
		\Comment modifica 1
\li $\attrib{x}{size} \gets \attrib{\attrib{x}{left}}{size} + \attrib{\attrib{x}{right}}{size} + 1$
		\Comment modifica 2
\end{codebox}