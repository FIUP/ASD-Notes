\section{Lezione del 14/03/2018}
\paragraph{Esercizio (importante)} 
\begin{align*}
	T(n) & = 2T(\frac{n}{2}) + 6n \\
		& = 2T(\frac{n}{2}) + \Theta (n) = \Theta (n \log n) \\
		\text{vale } \exists c & > 0 \ \exists n_0 : \forall n \geq n_0 \Rightarrow \Theta(n) \leq cn
\end{align*}
Voglio dimostrare che 
\begin{enumerate}
	\item $T(n) = O(n \log n)$
	\item $T(n) = \Omega (n \log n)$
\end{enumerate}

\subparagraph{1.} $T(n) = O(n \log n)$
\begin{displaymath}
	\text{significa che } \exists d > 0 \ \exists n_1 \in \mathbb{N} \ \vert \ T(n) \leq dn \log n \quad \forall n \geq n_1
\end{displaymath}
Dimostro per induzione $T(n) \leq dn \log n \quad \forall n \geq n_1$.\par
Ometto il caso base, poiché non è molto interessante (mi basterebbe aumentare ulteriormente $d$ per avere
un valore accettabile).
\begin{align*}
	T(n) & \leq 2T\Big(\frac{n}{2}\Big) + cn && \text{ip. induttiva } T\Big(\frac{n}{2}\Big) = d \frac{n}{2} \log \frac{n}{2} \\
	& \leq 2 \cdot \frac{n}{2} d \log \frac{n}{2} + cn && \Big(\log \frac{n}{2} = \log n - \log 2\Big) \\
	& = dn \log n - dn \log 2 + cn \\
	& = dn \log n - n(d \log 2 - c) \leq dn \log n \\
	& \Rightarrow - n(d \log 2 - c) \leq 0 \\
	& n(d \log 2 - c) \geq 0 \\
	& d \log 2 - c \geq 0 \\
	& \qquad d \geq \frac{c}{\log 2}
\end{align*}

\subparagraph{2.} $T(n) \geq dn \log n$ è analoga.

\paragraph{Esercizio} $T(n) = T\big(\frac{n}{3} \big) + T\big(\frac{2n}{3} \big) + \Theta (n)$
$\quad (\Theta (n) \leq c \cdot n)$\par
Ipotizzo un andamento simile a Merge Sort: $\Theta (n \log n)$. Dimostro:  
\begin{enumerate}
	\item $T(n) = O(n \log n)$
	\item $T(n) = \Omega (n \log n)$
\end{enumerate}

\subparagraph{1.} $\exists d > 0 : \forall n > n_0 \Rightarrow T(n) \leq dn \log n$.\par
Ometto il caso base. L'ipotesi induttiva è la seguente:
\[ 
	T(n) \leq d \frac{n}{3} \log \frac{n}{3} + d \frac{2n}{3} \log \frac{2n}{3} + cn
\]
Calcoli \dots
\begin{align*}
	T(n) & \leq T\Big(\frac{n}{3} \Big) + T\Big(\frac{2n}{3} \Big) + cn \\
	& \leq d \frac{n}{3} \log \frac{n}{3} + \frac{2n}{3} \log \frac{2n}{3} = \\
	& = \frac{dn}{3} \Big(\log n - \log 3 \Big) + d \frac{2n}{3} \Big(\log n - \log \frac{2}{3} \Big) + cn = \\
	& = dn \log n - \frac{dn}{3} \Big( \log 3 - 2 \log \frac{2}{3} \Big) + cn = \\
\end{align*}
\begin{align*}
	& = dn \log n - \frac{dn}{3} \Big( \log 3 - \log \frac{4}{9} \Big) + cn = \\
	& = dn \log n - n \Big( \frac{d}{3} \log \frac{27}{4} - c \Big) \leq dn \log n\\
	& \Rightarrow d \geq \frac{3c}{\log \frac{27}{4}} \qquad (\log \frac{27}{4} > 1 \text{ poiché } arg > 1)
\end{align*}

\subparagraph{2.} $T(n) = \Omega (n \log n)$ è analoga $\Rightarrow \text{vale } T(n) = \Theta (n \log n)$