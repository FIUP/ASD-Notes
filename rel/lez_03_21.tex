\section{Lezione del 21/03/2018}
\paragraph{Ordinamento} Finora abbiamo visto due algoritmi di ordinamento, in cui avevamo le seguenti 
premesse:
\begin{itemize}
	\item[] \texttt{IN}: $a_1\dots a_n$;
	\item[] \texttt{OUT}: permutazione $a'_1\dots a'_n$ ordinata.
\end{itemize}
In particolare, abbiamo concluso che:
\begin{itemize}
	\item \texttt{Insertion Sort}: $O(n^2)$, basato su scambi;
	\item \texttt{Merge Sort}: $\Theta(n \log n)$, ma con un costo in termini di \emph{memoria}.
\end{itemize}

\paragraph{Memoria} 
\begin{itemize}
	\item \texttt{Insertion Sort}: 
	$$input \ + \ 1 \text{ variabile} \Rightarrow 
		\text{spazio \emph{costante} } \Theta (1) \text{ (detto ``in loco'')}$$
	\item \texttt{Merge Sort}: spazio con costo lineare.
	\begin{align*}
		S_{MS}(n) & = max\Big{\{} S\Big( \Big{\lfloor}{\frac{n}{2}}\Big{\rfloor} \Big), 
		\ S\Big( \Big{\lceil}{\frac{n}{2}}\Big{\rceil} \Big), \ \Theta (n) \Big{\}} \\
		& = \Theta (n)
	\end{align*}
\end{itemize}

\subsection{Heapsort}
L'\href{https://en.wikipedia.org/wiki/Heapsort}{Heapsort}\footnote{Anche qui, si consiglia di dare %
un occhio ad altre fonti. In classe, sono stati viste molte rappresentazioni grafiche degli heap, %
e, come già detto, in \LaTeX $\ $non è per me facile rappresentarli.} è un algoritmo di ordinamento 
basato su 
una struttura chiamata \href{https://en.wikipedia.org/wiki/Heap_(data_structure)}{heap}, che prende le caratteristiche positive di 
\texttt{Insertion Sort} e \texttt{Merge Sort}:
\begin{itemize}[noitemsep]
	\item in ``loco'' (spazio $\Theta (1)$);
	\item complessità $\Theta(n \log n)$.
\end{itemize}

\paragraph{Cos'è un heap?} Un \texttt{heap} è una struttura dati basata sugli alberi che
soddisfa la ``proprietà di heap'': se A è un genitore di B, allora la chiave di A è ordinata rispetto
alla chiave di B conformemente alla relazione d'ordine applicata all'intero heap.\par
Seguono alcune definizioni.
\begin{description}
	\item[Altezza:] è la distanza dalla radice alla foglia più distante;
	\item[Albero completo:] è un albero di altezza $h$ con $\displaystyle\sum_{i=0}^h 2^i - 1$ nodi;
	\item[Albero quasi completo:] è un albero completo a tutti i livelli eccetto l'ultimo, in cui 
	possono mancare delle foglie.
\end{description}

Gli heap verranno rappresentati in array monodimensionali, nel modo descritto di seguito:
$$\forall i > 0 $$
\begin{itemize}
	\item \texttt{A[i]} è il nodo genitore;
	\item \texttt{A[2i]} è il figlio sx del nodo \texttt{A[i]};
	\item \texttt{A[2i+1]} è il figlio dx.
\end{itemize}
Inoltre, ogni array \texttt{A} sarà dinamico, e avrà:
\begin{itemize}
	\item $A.length$ potenziale spazio, capacità massima dell'array;
	\item $A.heapsize$ celle effettive dell'array.
\end{itemize}

Vediamo alcune funzioni di utilità che verranno usate.
\begin{codebox}
$\Procname{\proc{Left}(i)}$
\zi \Comment restituisce il figlio sx del nodo $i$
\li	\Return $2*i$
\end{codebox}
\begin{codebox}
$\Procname{\proc{Right}(i)}$
\zi \Comment restituisce il figlio dx del nodo $i$
\li	\Return $2*i+1$
\end{codebox}
\begin{codebox}
$\Procname{\proc{Parent}(i)}$
\zi \Comment restituisce il genitore del nodo $i$
\li	\Return $\floor{i/2}$
\end{codebox}

\subsubsection{Max Heap}
\emph{Max Heap} è uno heap che soddisfa la seguente proprietà:
\begin{gather*}
	\forall \text{ nodo } A[i], \\
	A[i] \geq \text{discendenti} \\
	\Downarrow \\
	A[i] \geq A[\text{\emph{Left(i)}}], \ A[\text{\emph{Right(i)}}]
\end{gather*}
\paragraph{Osservazioni}
\begin{itemize}[label=$\bullet$]
	\item Uno heap con un solo elemento è un \emph{Max Heap}.
	\item Dati due Max Heap $T_1$ e $T_2$ e un nodo $N$, possiamo ``combinarli'' in uno heap con
	$N$ come radice, $T_1$ come \emph{left} e $T_2$ come \emph{right}.
\end{itemize}

Ecco ora una procedura che, dato un nodo $i$, trasforma in un Max Heap il sotto-albero eradicato in esso (con radie $i$).

\begin{codebox}
\Procname{\proc{MaxHeapify}$(A,i)$}
\li $l \gets \proc{Left}(i)$
\li $r \gets \proc{Right}(i)$
\li \If $(l \leq \attrib{A}{heapsize}) \kw{ and } (A[l] > A[i])$
\li 	\Then
			$\id{max} \gets l$
		\Else 
\li			$\id{max} \gets i$			
		\End
\li \If $(\id{max} \neq i)$
\li 	\Then 
			$A[i] \leftrightarrow A[\id{max}]$
\li			$\proc{MaxHeapify}(A,\id{max})$
\end{codebox}

L'algoritmo ha un costo di $O(h)$, con $h$ altezza del sotto-albero radicato in $i$, con 
$$O(h) \cong O(\log n) \qquad \text{(Omessa la dimostrazione)}$$
Ora vogliamo scrivere una procedura che costruisce un \emph{Max Heap} da un array qualunque. \par
Quali sono i nodi foglia?
\begin{itemize}
	\item Se $i \geq \floor{\frac{n}{2}} + 1$
	\begin{align*}
		2i & = 2\Big( \frac{n}{2} + 1 \Big) \geq n + 2 - 1 = n + 1 \\
		& \Rightarrow i \text{ foglia}
	\end{align*}
	\item Se $i \leq \floor{\frac{n}{2}}$
	\begin{align*}
		2i & = 2 \floor{\frac{n}{2}} \leq n \\
		& \Rightarrow i \text{ non foglia}
	\end{align*}
\end{itemize}

\begin{codebox}
\Procname{\proc{BuildMaxHeap}$(A)$}
\li $\attrib{A}{heapsize} = \attrib{A}{length}$
\li \For $i \gets \floor{\attrib{A}{length}/2} \kw{ down} \To 1$
\li 	\Do
			$\proc{MaxHeapify}(A,i)$
		\End
\end{codebox}