\section{Lezione del 21/03/2018}
\paragraph{Ordinamento} Finora abbiamo visto due algoritmi di ordinamento, in cui avevamo le seguenti 
premesse:
\begin{itemize}
	\item[] \texttt{IN}: $a_1\dots a_n$;
	\item[] \texttt{OUT}: permutazione $a'_1\dots a'_n$ ordinata.
\end{itemize}
In particolare, abbiamo concluso che:
\begin{itemize}
	\item \texttt{Insertion Sort}: $O(n^2)$, basato su scambi;
	\item \texttt{Merge Sort}: $\Theta(n \log n)$, ma con un costo in termini di \emph{memoria}.
\end{itemize}

\paragraph{Memoria} 
\begin{itemize}
	\item \texttt{Insertion Sort}: 
	$$input \ + \ 1 \text{ variabile} \Rightarrow 
		\text{spazio \emph{costante} } \Theta (1) \text{ (detto ``in loco'')}$$
	\item \texttt{Merge Sort}: spazio con costo lineare.
	\begin{align*}
		S_{MS}(n) & = max\Big{\{} S\Big( \Big{\lfloor}{\frac{n}{2}}\Big{\rfloor} \Big), 
		\ S\Big( \Big{\lceil}{\frac{n}{2}}\Big{\rceil} \Big), \ \Theta (n) \Big{\}} \\
		& = \Theta (n)
	\end{align*}
\end{itemize}

\subsection{Heapsort} \label{heapsort} 
L'\href{https://en.wikipedia.org/wiki/Heapsort}{Heapsort}\footnote{Anche qui, si consiglia di dare %
un occhio ad altre fonti. In classe, sono stati viste molte rappresentazioni grafiche degli heap, %
e, come già detto, in \LaTeX $\ $non è per me facile rappresentarli.} è un algoritmo di ordinamento 
basato su 
una struttura chiamata \href{https://en.wikipedia.org/wiki/Heap_(data_structure)}{heap}, che prende le caratteristiche positive di 
\texttt{Insertion Sort} e \texttt{Merge Sort}:
\begin{itemize}[noitemsep]
	\item in ``loco'' (spazio $\Theta (1)$);
	\item complessità $\Theta(n \log n)$.
\end{itemize}

\paragraph{Cos'è un heap?} Un \texttt{heap} è una struttura dati basata sugli alberi che
soddisfa la ``proprietà di heap'': se A è un genitore di B, allora la chiave di A è ordinata rispetto
alla chiave di B conformemente alla relazione d'ordine applicata all'intero heap.\par
Seguono alcune definizioni.
\begin{description}
	\item[Altezza:] è la distanza dalla radice alla foglia più distante;
	\item[Albero completo:] è un albero di altezza $h$ con $\displaystyle\sum_{i=0}^h 2^i - 1$ nodi;
	\item[Albero quasi completo:] è un albero completo a tutti i livelli eccetto l'ultimo, in cui 
	possono mancare delle foglie.
\end{description}

Gli heap verranno rappresentati in array monodimensionali, nel modo descritto di seguito:
$$\forall i > 0 $$
\begin{itemize}
	\item \texttt{A[i]} è il nodo genitore;
	\item \texttt{A[2i]} è il figlio sx del nodo \texttt{A[i]};
	\item \texttt{A[2i+1]} è il figlio dx.
\end{itemize}
Inoltre, ogni array \texttt{A} sarà dinamico, e avrà:
\begin{itemize}
	\item $A.length$ potenziale spazio, capacità massima dell'array;
	\item $A.heapsize$ celle effettive dell'array.
\end{itemize}

Vediamo alcune funzioni di utilità che verranno usate.
\begin{codebox}
$\Procname{\proc{Left}(i)}$
\zi \Comment restituisce il figlio sx del nodo $i$
\li	\Return $2*i$
\end{codebox}
\begin{codebox}
$\Procname{\proc{Right}(i)}$
\zi \Comment restituisce il figlio dx del nodo $i$
\li	\Return $2*i+1$
\end{codebox}
\begin{codebox}
$\Procname{\proc{Parent}(i)}$
\zi \Comment restituisce il genitore del nodo $i$
\li	\Return $\floor{i/2}$
\end{codebox}

\subsubsection{Max Heap}
\emph{Max Heap} è uno heap che soddisfa la seguente proprietà:
\begin{gather*}
	\forall \text{ nodo } A[i], \\
	A[i] \geq \text{discendenti} \\
	\Downarrow \\
	A[i] \geq A[\text{\emph{Left(i)}}], \ A[\text{\emph{Right(i)}}]
\end{gather*}
\paragraph{Osservazioni}
\begin{itemize}[label=$\bullet$]
	\item Uno heap con un solo elemento è un \emph{Max Heap}.
	\item Dati due Max Heap $T_1$ e $T_2$ e un nodo $N$, possiamo ``combinarli'' in uno heap con
	$N$ come radice, $T_1$ come \emph{left} e $T_2$ come \emph{right}.
\end{itemize}

Ecco ora una procedura che, dato un nodo $i$, trasforma in un Max Heap il sotto-albero eradicato in esso (con radie $i$).

\begin{codebox}
\Procname{\proc{MaxHeapify}$(A,i)$}
\li $l \gets \proc{Left}(i)$
\li $r \gets \proc{Right}(i)$
\li \If $(l \leq \attrib{A}{heapsize}) \kw{ and } (A[l] > A[i])$
\li 	\Then
			$\id{max} \gets l$
		\Else 
\li			$\id{max} \gets i$			
		\End
\li \If $(\id{max} \neq i)$
\li 	\Then 
			$A[i] \leftrightarrow A[\id{max}]$
\li			$\proc{MaxHeapify}(A,\id{max})$
\end{codebox}

L'algoritmo ha un costo di $O(h)$, con $h$ altezza del sotto-albero radicato in $i$, con 
$$O(h) \cong O(\log n) \qquad \text{(omessa la dimostrazione)}$$
Ora vogliamo scrivere una procedura che costruisce un \emph{Max Heap} da un array qualunque. \par
Quali sono i nodi foglia?
\begin{itemize}
	\item Se $i \geq \floor{\frac{n}{2}} + 1$
	\begin{align*}
		2i & = 2\Big( \frac{n}{2} + 1 \Big) \geq n + 2 - 1 = n + 1 \\
		& \Rightarrow i \text{ foglia}
	\end{align*}
	\item Se $i \leq \floor{\frac{n}{2}}$
	\begin{align*}
		2i & = 2 \floor{\frac{n}{2}} \leq n \\
		& \Rightarrow i \text{ non foglia}
	\end{align*}
\end{itemize}

\begin{codebox}
\Procname{\proc{BuildMaxHeap}$(A)$}
\li $\attrib{A}{heapsize} = \attrib{A}{length}$
\li \For $i \gets \floor{\attrib{A}{length}/2} \kw{ down} \To 1$
\li 	\Do
			$\proc{MaxHeapify}(A,i)$
		\End
\end{codebox}
L'algoritmo esegue $\frac{n}{2}$ volte \texttt{MaxHeapify}, che ha un costo di $O(n \log n)$, 
tuttavia questa stima è molto pessimistica.\par
Definiamo: 
\begin{itemize}[noitemsep]
	\item $h_T$ altezza del cammino più lungo dello heap;
	\item $h_T - 1$ di conseguenza è l'altezza dell'albero meno l'ultimo livello, che è
	generalmente incompleto.
\end{itemize}

\begin{align*}
	n & = \Big( 2^{(h_T-1)+1} - 1\Big) + 1 \\
	& = 2^{h_T} \\
	& h_T \leq \log n \\
	& n \geq 2^{h_T}
\end{align*}
\begin{align*}
	T(n) & = \displaystyle\sum_{h = 1}^{\floor{\log n}} 2^{h_T - h} \cdot O(h) \\
	& \hspace{1.4cm} (2^{h_T - h} = \text{\# chiamate a \texttt{MaxHeapify} al livello } h) \\
	& = \displaystyle\sum_{h = 1}^{\floor{\log n}} \frac{2^{h_T}}{2^h}O(h) \qquad (2^{h_T} = n) \\
	& = O \Big( \big(\displaystyle\sum_{h = 1}^{\floor{\log n}} \frac{h}{2^h} \big) n \Big) = O(n) 
		\qquad \Big( \displaystyle\sum_{h = 1}^{\floor{\log n}} \frac{h}{2^h} \leq \displaystyle\sum_{h = 1}^{\infty} \frac{h}{2^h} 
		= \frac{\frac{1}{2}}{(1 - \frac{1}{2})^2} = 2 \Big)
\end{align*}

%22/03/2018%

Passiamo ora all'algoritmo di ordinamento \texttt{Heapsort}. La radice di un 
\emph{Max Heap} contiene il valore massimo. Quindi, la prima operazione, e quella su cui si basa \texttt{Heapsort}, 
consiste nel mettere la radice in ultima posizione.
\begin{align*}
\text{Es. \texttt{A}: }& \text{9 8 7 5 7 4 0 4 3 6 1 2 \hspace{1cm}} \text{è un max heap.} \\
\Rightarrow & \text{8 7 5 7 4 0 4 3 6 1 2 9} \hspace{1.2cm} \text{ignoro l'ultimo elemento, chiamo} \\
& \hspace{5.2cm}\text{\texttt{MaxHeapify} sulla radice e itero.}
\end{align*}

Poi chiama \texttt{MaxHeapify} sul resto dell'array per renderlo un \emph{Max Heap}, e itera il procedimento 
sul nuovo array.

\begin{codebox}
\Procname{\proc{HeapSort}$(A)$}
\li \proc{BuildMaxHeap}(A) 
    \Comment $O(n)$
\li \For $i \gets \attrib{A}{length}$ \kw{down} \To $2$
\li     \Do
            $A[1] \leftrightarrow A[i]$
\li         $\attrib{A}{heapsize} \gets \attrib{A}{heapsize} - 1 $
\li         $\proc{MaxHeapify}(A,1)$
                \Comment $O(\log n)$
        \End
\end{codebox}

\paragraph{Costo?} Il costo complessivo è di $O(n \log n)$.

\subsection{Code con priorità} \label{codeconpriorita}
$S$ insieme dinamico di oggetti. \par
$x$ è l'indice, $\attrib{x}{key}$ è il corrispondente valore relativo a quell'indice.
Voglio poter eseguire le seguenti operazioni:
\begin{itemize}[noitemsep]
	\item \texttt{Insert(S, x)}
	\item \texttt{Max(S)}
	\item \texttt{ExtractMax(S)}
	\item \texttt{IncreaseKey(S, x, $\Delta$)}
	\item \texttt{ChangeKey(S, x, $\Delta$)}
	\item \texttt{Delete(S, x)}
\end{itemize}
\paragraph{Idea} Uso un \emph{Max Heap} (A).

\begin{codebox}
\Procname{\proc{Max}$(A)$}
\li \If $\attrib{A}{heapsize} = 0$
\li     \Then
            \kw{error}
\li     \Else
            \Return $A[1]$
        \End
\end{codebox}
La procedura \texttt{Max(A)} ha complessità costante $\Theta (1)$.

\begin{codebox}
\Procname{\proc{ExtractMax}$(A)$}
\li $\id{max} \gets A[1]$
\li $A[1] \gets A[\attrib{A}{heapsize}]$
\li $\attrib{A}{heapsize} \gets \attrib{A}{heapsize} - 1$
\li $\proc{MaxHeapify}(A,1)$
        \Comment ripristina le proprietà di \emph{MaxHeap}
\li \Return $\id{max}$
\end{codebox}
La procedura \texttt{ExtractMax(A)} ha la stessa complessità di \texttt{MaxHeapify}: \\
$O(\log n)$.
\bigskip

Per \texttt{Insert}, le cose diventano più delicate. L'idea è quella di inserire
in coda ad A: in questo modo, l'unico elemento che potrebbe compromettere la proprietà di 
\emph{Max Heap} è la cella di indice $i$ (nel nostro caso, l'ultima). Deve vale la proprietà:
\begin{gather*}
	\text{Per ogni } j \neq i \\
	A[j] \leq \text{antenati}
\end{gather*}
Non possiamo dire nulla su $i$. Va ristabilita la proprietà di \emph{Max Heap}: per fare ciò 
usiamo la procedura \texttt{MaxHeapifyUp}.
\begin{codebox}
\Procname{\proc{MaxHeapifyUp}$(A,i)$}
\li \If $(i > 1) \kw{ and } (A[i] > A[\proc{Parent}(i)])$
\li     \Then
            $A[i] \leftrightarrow A[\proc{Parent}(i)]$
\li         $\proc{MaxHeapifyUp}(A,\proc{Parent}(i))$
        \End
\end{codebox}

\paragraph{Correttezza di MaxHeapifyUp}
\subparagraph{Casi base} 
\begin{itemize}
	\item[] \texttt{(i = 1)} ok, non faccio nulla;
	\item[] \texttt{(A[i] $\leq$ A[Parent(i)])} ok, la proprietà di \emph{Max Heap} è mantenuta.
\end{itemize}
\subparagraph{Induzione}
\begin{itemize}
	\item[] \texttt{(A[i] > A[Parent(i)])} scambio le due celle. I discendenti (sotto-alberi) della nuova
	cella \texttt{A[i]} mantengono la proprietà di \emph{Max Heap}.
\end{itemize}

\paragraph{Costo?} $O(\log i)$, nel caso peggiore $O(\log n)$.
\bigskip

Ecco ora lo pseudocodice della funzione \texttt{Insert}.
\begin{codebox}
\Procname{\proc{Insert}$(A,x)$}
\li $\attrib{A}{heapsize} = \attrib{A}{heapsize} + 1$
\li $A[\attrib{A}{heapsize}] \gets x$
\li $\proc{MaxHeapifyUp}(A,\attrib{A}{heapsize})$
\end{codebox}
\texttt{Insert} ha costo $O(\log n)$, lo stesso di \texttt{MaxHeapifyUp}.

\begin{codebox}
\Procname{\proc{IncreaseKey}$(A,i,\delta)$}
\zi \Comment \emph{Precondizione}: $\delta \geq 0$
\li $A[i] \gets A[i] + \delta$
\li $\proc{MaxHeapifyUp}(A,i)$
\end{codebox}
\texttt{IncreaseKey} ha costo $O(\log n)$.

\begin{codebox}
\Procname{\proc{ChangeKey}$(A,i,\delta)$}
\li $A[i] \gets A[i] + \delta$
\li \If $\delta > 0$
\li     \Then
            $\proc{MaxHeapifyUp}(A,i)$
\li     \Else
        \Comment $\delta \leq 0$
\li         $\proc{MaxHeapify}(A,i)$
        \End
\end{codebox} 
\texttt{ChangeKey} è come \texttt{IncreaseKey}, ma può utilizzare valori di $\delta$ qualsiasi, ed è
corretto per la seguente proprietà:
\begin{gather*}
	\text{Se per ogni } j \neq i \quad A[j] \geq \text{discendenti} \\
	\Rightarrow \text{dopo \texttt{MaxHeapify} ho un \emph{MaxHeap}}
\end{gather*}

\begin{codebox}
\Procname{\proc{DeleteKey}$(A,i)$}
\li $\id{old} \gets A[i]$
\li $A[i] \gets A[\attrib{A}{heapsize}]$
\li $\attrib{A}{heapsize} \gets \attrib{A}{heapsize} - 1$
\li \If $\id{old} \leq A[i]$
\li     \Then
            $\proc{MaxHeapifyUp}(A,i)$
\li     \Else
\li         $\proc{MaxHeapify}(A,i)$
        \End
\end{codebox}
\texttt{DeleteKey} ha costo $O(\log n)$.