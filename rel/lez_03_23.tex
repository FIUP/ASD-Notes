\section{Lezione del 23/03/2018}

\subsection{Quicksort}
Il \href{https://en.wikipedia.org/wiki/Quicksort}{Quicksort} è probabilmente l'algoritmo di ordinamento
più utilizzato e nella pratica efficiente, nonostante abbia un caso pessimo di $O(n^2)$.
\begin{itemize}
	\item Caso pessimo $O(n^2)$;
	\item Caso medio e migliore $O(n \log n)$;
	\item costanti basse.
\end{itemize}
Si basa sul paradigma del \textit{divide et impera}:
\begin{itemize}
	\item \textit{Divide}
	\begin{itemize}[label=$\rightarrow$]
		\item Secglie un \textit{pivot} $x$ in \texttt{A[p, r]};
		\item partiziona in \texttt{A[p, q-1] $\leq x$} e \texttt{A[q+1, r] $\geq x$};
	\end{itemize}
	\item \textit{Impera} \par
	Ricorre su \texttt{A[p, q-1]} e \texttt{A[q+1, r]};
	\item \textit{Combina}\par
	(Non fa nulla).
\end{itemize}

\paragraph{Pseudocodice} Segue lo pseudocodice del \texttt{Quicksort}.
\begin{codebox}
\Procname{\proc{Quicksort}$(A,p,r)$}
\li \If $p < r$
\li 	\Then
			$q \gets \proc{Partition}(A,p,r)$
\li 		$\proc{Quicksort}(A,p,q)$
\li 		$\proc{Quicksort}(A,q+1,r)$
		\End
\end{codebox}
\begin{codebox}
\Procname{\proc{Partition}$(A,p,r)$}
\li $x \gets A[r]$ 
	\Comment pivot \texttt{A[r]}
\li $i \gets p - 1$
\zi \Comment \texttt{A[p, i]} $\leq x$
\zi \Comment \texttt{A[i+1, j-1]} $> x$
\li \For $j \gets p$ \To $r-1$
\li 	\Do
			\If $A[j] \leq x$
\li 			\Then
					$i \gets i + 1$
\li 				$A[i] \leftrightarrow A[j]$
				\End
		\End
\li $A[i+1] \leftrightarrow A[r]$
\li \Return $i + 1$
\end{codebox}

\subsubsection{Correttezza di Quicksort(A, p, r)}
\begin{description}
	\item[Caso base] array già ordinato, 0 o 1 elemento.
	\item[Induzione] Abbiamo, dopo \texttt{Partition}
	\begin{center}
		\begin{tabular}{| l | l | l |}
			\hline
			$\leq$ \texttt{A[q]} & \texttt{A[q]} & $\geq$ \texttt{A[q]} \\
			\hline 
		\end{tabular}
	\end{center}
	\texttt{Quicksort(A, p, q)} 
	\begin{tabular}{|l|l|l|}
		\hline
		$\leq$ \texttt{A[q]}, ord & \texttt{A[q]} & $>$ \texttt{A[q]} \\
		\hline
	\end{tabular} \par
	\texttt{Quicksort(A, q+1, r)} 
	\begin{tabular}{|l|l|l|}
		\hline
		$\leq$ \texttt{A[q]}, ord & \texttt{A[q]} & $>$ \texttt{A[q]}, ord \\
		\hline
	\end{tabular}
\end{description}

\paragraph{Esempio}

Dato l'array $A$, scelgo come \emph{pivot} $x$ l'ultimo elemento.

\begin{center}
	\begin{tabular}{|l|l|l|l|l|l|}
		\hline
		9 & 6 & 0 & 8 & 4 & 2 \\
		\hline
	\end{tabular}
	\hspace{1cm}
	\emph{pivot: }
	\begin{tabular}{|l|}
		\hline
		2 \\
		\hline
	\end{tabular}
\end{center}

\texttt{i} punta alla cella 0 (ossia nessuna cella) \par
\texttt{j} punta alla cella 1:
\begin{tabular}{|l|}
	\hline
	9 \\
	\hline
\end{tabular}

$9 > 2$? Sì $\Rightarrow$ \texttt{j++} \par
$6 > 2$? Sì $\Rightarrow$ \texttt{j++} \par
$0 > 2$? No $\Rightarrow$ \texttt{i++}, \texttt{A[i]} $\leftrightarrow$ \texttt{A[j]}, \texttt{j++} \par

\begin{center}
	\begin{tabular}{|l|l|l|l|l|l|}
		\hline
		0 & 6 & 9 & 8 & 4 & 2 \\
		\hline
	\end{tabular}
	\hspace{1cm}
	\emph{pivot: }
	\begin{tabular}{|l|}
		\hline
		2 \\
		\hline
	\end{tabular}
\end{center}

\texttt{i} punta alla cella 1: 
\begin{tabular}{|l|}
	\hline
	0 \\
	\hline
\end{tabular} \par
\texttt{j} punta alla cella 4:
\begin{tabular}{|l|}
	\hline
	8 \\
	\hline
\end{tabular}

$8 > 2$? Sì $\Rightarrow$ \texttt{j++} \par
$4 > 2$? Sì $\Rightarrow$ \texttt{j++} \par
Scambio \texttt{A[i+1]} con \texttt{x}, ottenendo \par

\begin{center}
	\begin{tabular}{|l|l|l|l|l|l|}
		\hline
		0 & 2 & 9 & 8 & 4 & 6 \\
		\hline
	\end{tabular}
\end{center}

I primi due ($i+1$) elementi sono ordinati: 
\begin{center}
	\begin{tabular}{|l|l|}
		\hline
		0 & 2 \\
		\hline
	\end{tabular}
\end{center}

Chiamo ricorsivamente \texttt{Quicksort} con $q = i + 1$.

\begin{center}
	\begin{tabular}{|l|l|l|l|}
		\hline
		9 & 8 & 4 & 6 \\
		\hline
	\end{tabular}
	\hspace{1cm}
	\emph{pivot: }
	\begin{tabular}{|l|}
		\hline
		6 \\
		\hline
	\end{tabular}
\end{center}

\texttt{i} punta alla cella 0 (ossia nessuna cella) \par
\texttt{j} punta alla cella 1:
\begin{tabular}{|l|}
	\hline
	9 \\
	\hline
\end{tabular}

$9 > 6$? Sì $\Rightarrow$ \texttt{j++} \par
$8 > 6$? Sì $\Rightarrow$ \texttt{j++} \par
$4 > 6$? No $\Rightarrow$ \texttt{i++}, \texttt{A[i]} $\leftrightarrow$ \texttt{A[j]}, \texttt{j++} \par

\begin{center}
	\begin{tabular}{|l|l|l|l|}
		\hline
		4 & 8 & 9 & 6 \\
		\hline
	\end{tabular}
	\hspace{1cm}
	\emph{pivot: }
	\begin{tabular}{|l|}
		\hline
		2 \\
		\hline
	\end{tabular}
\end{center}

\texttt{i} punta alla cella 1: 
\begin{tabular}{|l|}
	\hline
	4 \\
	\hline
\end{tabular} \par
\texttt{j} punta alla cella 4:
\begin{tabular}{|l|}
	\hline
	6 \\
	\hline
\end{tabular}, quindi ho finito. \par
Scambio \texttt{A[i+1]} con \texttt{x}, ottenendo \par
\begin{center}
	\begin{tabular}{|l|l|l|l|}
		\hline
		4 & 6 & 9 & 8 \\
		\hline
	\end{tabular}
\end{center}

I primi due ($i+1$) elementi sono ordinati: 
\begin{center}
	\begin{tabular}{|l|l|}
		\hline
		4 & 6 \\
		\hline
	\end{tabular}
\end{center}

Chiamo ricorsivamente \texttt{Quicksort} con $q = i + 1$.

\begin{center}
	\begin{tabular}{|l|l|}
		\hline
		9 & 8 \\
		\hline
	\end{tabular}
	\hspace{1cm}
	\emph{pivot: }
	\begin{tabular}{|l|}
		\hline
		8 \\
		\hline
	\end{tabular}
\end{center}

\texttt{i} punta alla cella 0 (ossia nessuna cella) \par
\texttt{j} punta alla cella 1:
\begin{tabular}{|l|}
	\hline
	9 \\
	\hline
\end{tabular}

$9 > 8$? Sì $\Rightarrow$ \texttt{j++} \par
Ho finito, scambio \texttt{A[i+1]} con \texttt{x}, ottenendo \par
\begin{center}
	\begin{tabular}{|l|l|}
		\hline
		8 & 9 \\
		\hline
	\end{tabular}
\end{center}

Guardando l'array completo ottengo il risultato atteso:
\begin{center}
	\begin{tabular}{|l|l|l|l|l|l|}
		\hline
		0 & 2 & 4 & 6 & 8 & 9 \\
		\hline
	\end{tabular}
\end{center}

\subsubsection{Complessità di Quicksort}

\texttt{Partition} costa $\Theta (n)$
\begin{align*}
	T^{QS} = \Theta (n) + T^{QS}(q - p) \ + \ &T^{QS}(n - (q - p) - 1) \\
	& q - p < n
\end{align*}

\paragraph{Caso peggiore}
\begin{align*}
	T^{QS} = & \Theta (n) + T^{QS}(n - 1) = \Theta (n^2) \qquad (\Theta (n) = cn)\\
	& T(n) \\
	& cn \\
	& cn-1 \\
	& cn-2 \\
	&\dots \\
	& d
\end{align*}
\begin{align*}
	\displaystyle\sum^{n-1}_{j=1} c(n-j) + d & = \displaystyle\sum^{n}_{k=1} ck + d = \\
	& = c\displaystyle\sum^{n}_{k=1}k + d \qquad \left(\frac{c(n+1)n}{2} + d = \Theta (n^2) \right)\\
\end{align*}
\[ T(n) = \Theta(n^2) \Rightarrow
	\begin{cases}
	= O(n^2) \\
	= \Omega(n^2)
	\end{cases}
\]

\begin{itemize}[label=$\bullet$]
	\item $T(n) = O(n^2)$
	\begin{align*}
		T(n) = \ & O(n^2) \Rightarrow T(n) \leq cn^2 \qquad \forall n \geq n_0, \; c > 0 \\
		= \ & T(n - 1) + \Theta(n) \leq dn \\
		\leq \ & c(n - 1) + dn \\
		= \ & cn^2 - 2cn + c + dn \leq cn^2 \\
		& 2cn - dn - c \geq 0 \\
		& n (2c - d) - c \geq 0 \qquad \text{ok, } c > \frac{d}{2} \\
		T(n) = \ & O(n^2)
	\end{align*}

	\item $T(n) = \Omega(n^2)$ analogo.
\end{itemize}

\paragraph{Caso migliore}
$$T(n) = 2 T\left( \frac{n}{2} \right) + \Theta (n) = \Theta(n \log n)$$

\paragraph{Caso medio} Qualunque partizionamento proporzionale da complessità \\ 
$\Theta (n \log n)$, come ad esempio
$$T(n) = T \left( \frac{n}{10} \right) + T \left( \frac{9n}{10} \right) + \Theta (n) = \Theta (n \log n)$$

Solo il caso in cui una delle due partizioni è costante, si ricade nel caso pessimo. Per ovviare al
problema, si può utilizzare una versione di \texttt{Partition} che rende impossibile il partizionamento costante.

\begin{codebox}
\Procname{\proc{RandomizedPartition}$(A,p,r)$}
\li $q \gets \proc{Random}(p,r)$ 
\li $A[q] \leftrightarrow A[r]$
\li \Return \proc{Partition}$(A,p,r)$
\end{codebox}
