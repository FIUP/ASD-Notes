\subsection{Radix Sort}
Il \href{https://en.wikipedia.org/wiki/Radix_sort}{Radix Sort} è un algoritmo di ordinamento in tempo
lineare $O(n)$, come \texttt{CountingSort}, che risolve i problemi di memoria di quest'ultimo.

L'idea è quella di ordinare cifra per cifra, dalla cifra meno significativa alla più significativa con un 
algoritmo \emph{stabile}.
\begin{align*}
	\text{(iniziale)} && \text{(terza cifra)} && \text{(seconda cifra)} && \text{(prima cifra)} \\
	329 && 720 && 720 && 329 \\
	457 && 355 && 329 && 355 \\
	657 && 436 && 436 && 436 \\
	839 && 457 && 839 && 457 \\
	436 && 657 && 355 && 657 \\
	720 && 329 && 457 && 720 \\
	355 && 839 && 657 && 839 \\
\end{align*}

\texttt{Input}: \texttt{A[1$\twodots$n]} con \texttt{A[i]} di $d$ cifre e base $b$,
\texttt{A[i]}$= a_d a_{d-1}\dots a_1$.

\begin{codebox}
\Procname{\proc{RadixSort}$(A,d)$}
\li \For $j \gets 1$ \To $d$
\li \Then
		$\text{ordina $A$ rispetto alla cifra $j$}$
		\Comment $A^j[i] = a_j a_{j-1} \dots a_1$
\zi 	$\text{con}$ \proc{CountingSort}
		\Comment $A^{j-1} ordinato$
	\End
\end{codebox}

\paragraph{Correttezza di RadixSort(A, d)}
\begin{itemize}
	\item \textbf{Inizializzazione}: ok;
	\item \textbf{Mantenimento}: se $A^{j-1}$ è ordinato e ordino rispetto alla $j$-esima cifra con
	un algoritmo stabile, allora $A^j$ è ordinato.
	$$i < i' \Rightarrow A^j[i] \leq A^j[i']$$
	\begin{align*}
		\text{Siano } & A^j[i] = a_j a_{j-1} \dots a_1 \\
		& A^j[i'] = a'_j a'_{j-1} \dots a_1
	\end{align*}
	
	Posso distinguere due casi:
	\begin{enumerate}
		\item \begin{align*}
			a_j \neq a'_j & \Rightarrow a_j < a'_j \\
			& \Rightarrow A^j[i] < A^j[i']
		\end{align*}
		\item \begin{align*}
			a_j = a'_j & \Rightarrow A^j[i] \leq A^j[i'] \qquad \text{(stabilità)}\\
			& \Rightarrow A^j[i] = a_j A^j[i] \leq \\
			& \leq A^j[i'] = a'_j A^{j-1}[i']
		\end{align*}
	\end{enumerate}
\end{itemize}

\paragraph{Complessità} 
\begin{gather*}
	d \text{ volte \texttt{CountingSort} } \; \Theta(n+b) \Rightarrow \Theta\big(d(n+b)\big) = \Theta(n) \\
	\text{con } d \text{ cifre } = \Theta(1), \text{ base } b = \Theta(n) 
\end{gather*}
\begin{align*}
	m \text{ bit, } r \text{ bit per cifra, } \frac{m}{r} \text{ cifre, base } 2^r \\
	\Theta( \frac{m}{r} (m + 2^r)) & = && r = \log_2 n \\
	& = \Theta( \frac{m}{ \log n} (n + 2^{ \log n} )) \\
	& = \Theta( \frac{m}{ \log n} n) && m = O( \log n) \\
	& 	= \Theta(n)
\end{align*}

\section{Tabelle Hash}
\begin{gather*}
	U \text{ universo delle chiavi} \\
	U = \{ 0, 1, \dots, \abs{U} - 1 \} \\
	T[0 \twodots \abs{U} - 1] \text{ tabella hash}	
\end{gather*}
\[ T[k] \text{ contiene} \quad
	\begin{cases}
		\text{elemento } x \text{ con } x.key = k & \text{se c'è} \\
		\const{nil} & \text{altrimenti}
	\end{cases}
\]

\begin{codebox}
\Procname{\proc{Insert}$(T,x)$}
\li $T[\attrib{x}{\id{key}}] \gets x$
	\Comment $\Theta(1)$
\end{codebox}
\begin{codebox}
\Procname{\proc{Delete}$(T,x)$}
\li $T[\attrib{x}{\id{key}}] \gets nil$
	\Comment $\Theta(1)$
\end{codebox}
\begin{codebox}
\Procname{\proc{Search}$(k)$}
\li \Return $T[k]$
	\Comment $\Theta(1)$
\end{codebox}

\paragraph{Problema} e.g. consideriamo che la \emph{key} sia di 8 caratteri (e 8 bit per rappresentare
un carattere). Risulta molto costosa in termini di memoria la tabella hash.
\begin{gather*}
	2^8 \dots 2^8 \\
	(2^8)^8 = 2^{64} \cong 10^{19}
\end{gather*}

\paragraph{Idea} 
\begin{gather*}
	U = \{ 0, 1, \dots , \abs{U} - 1 \} \\
	T[0 \dots m-1] \qquad m << \abs{U}
\end{gather*}
La ``traduzione'' per ottenere $x.key$ da $x$ cosa comporta?
\begin{gather*}
	h \colon U \to \{ 0, 1, \dots , m-1 \} \text{ funzione di \emph{hashing}} \\
	n = \text{\# elementi memorizzati nella tabella } T \\
	m = \text{\# celle}
\end{gather*}
Se $n > m$, esisteranno $x_1, x_2 : h(x_1.key) = h(x_2.key) \Rightarrow$ conflitto \par
\bigskip 
Abbiamo \emph{due soluzioni}:
\begin{enumerate}
	\item \textbf{Chaining} (\ref{hash:chaining});
	\item \textbf{Open Addressing} (\ref{hash:openaddessing}).
\end{enumerate}