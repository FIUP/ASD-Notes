\subsection{Chaining}\label{hash:chaining}
Il \emph{Chaining} propone come soluzione quella di mettere sulla tabella liste dinamiche
di elementi, invece che singoli elementi, in modo che in caso si incorra in una cella
già occupata dopo un \emph{hashing}, l'elemento venga inserito in coda (o in testa) alla
lista.

\paragraph{Idea} \texttt{T[i]} $=$ lista elementi $x$ tali che $h(x.key) = i$

\begin{codebox}
\Procname{\proc{Insert}$(T,x)$}
\li	$\text{Inserisci $x$ nella lista } T[h(x.key)]$
	\Comment $O(1)$
\end{codebox}
\begin{codebox}
\Procname{\proc{Delete}$(T,x)$}
\li	$\text{Delete $x$ from } T[h(x.key)]$
	\Comment $O(1)$
\end{codebox}
\begin{codebox}
\Procname{\proc{Search}$(T,k)$}
\li	$\text{Cerca in } T[h(k)] \text{ un elemento } x \text{ tale che } \attrib{x}{\id{key}} = k$ 
	\Comment $O(n)$
\end{codebox}
\texttt{Search} ha una complessità di $O(n)$, e questo è inaccettabile.
\begin{gather*}
	n = \text{\# elementi inseriti} \\
	m = \text{dimensione di } T \\
	\alpha = \frac{n}{m} \quad \text{fattore di carico} \\
	\alpha \text{ può essere }<, \ = \text{ oppure } > \text{ di } 1
\end{gather*}

\subsubsection{Hashing uniforme semplice}
Ogni elemento di \emph{input} è ``mandato'' da $h$ con la stessa probabilità $\left( \frac{1}{m} \right)$
in una delle $m$ celle.

\paragraph{Caso medio} $\Theta(1 + \alpha)$, 1 è l'accesso alla tabella.\par
Consideriamo $n_1,n_2,\dots,n_{m-1}$ la lunghezza delle $m$ liste. La lunghezza attesa
di una lista è:
$$E[n_j] = \displaystyle\sum_{i = 1}^{n} \frac{1}{m} \cdot 1 = \frac{n}{m} = \alpha$$

\paragraph{Ricerca di una chiave} La chiave può essere:
\begin{itemize}
	\item \emph{Assente}. \texttt{Search(k)}, \texttt{k} non c'è.
	\begin{itemize}
		\item Calcolo \texttt{h(k)} $\rightarrow$ ($\Theta(1)$);
		\item Accedo a \texttt{T[h(k)] = j} $\rightarrow$ ($\Theta(1)$);
		\item Scorro $n_j$ elementi ($n_j = \alpha$) $\rightarrow (\Theta(\alpha)$).  
	\end{itemize}
	Nel complesso, ho $\Theta(1+\alpha)$

	\item \emph{Presente}. \texttt{Search(k)}, \texttt{k} presente.
	\begin{itemize}
		\item \texttt{h(k)} e \texttt{T[h(k)]}\par
			$$\text{Se } x_1, x_2,\dots x_n \text{ sono gli elementi inseriti}$$
		Costo della ricerca di $x_i$:
		\begin{align*}
			& 1 + \text{\# elementi} \quad x_j : j > 1, \ h(x_i.key) = h(x_j.key) \\
			& = 1 + \displaystyle\sum_{j=i+1}^n \left(prob \ h(x_i.key) = h(x_j.key)\right) \\
			& = 1 + \displaystyle\sum_{j=i+1}^n \frac{1}{m} = 1 + \frac{n-i}{m}
		\end{align*}
		\begin{align*}
			& \frac{1}{n} \displaystyle\sum_{i=1}^n \left( 1 + \frac{n-i}{m} \right) \\
			& = \frac{1}{n} \left( n + \displaystyle\sum_{i=1}^n \frac{n-i}{m} \right) 
				= \frac{1}{n} \left( n + \frac{1}{m} \displaystyle\sum_{z=0}^{n-1} z \right) \\
			& = 1 + \frac{1}{m \cdot n} \cdot \frac{n(n-1)}{2} = 1 + \frac{n}{2m} - \frac{1}{2m} \cdot \left(\frac{n}{n} \right) && \alpha = \frac{n}{m} \\
			& = 1 + \frac{\alpha}{2} - \frac{\alpha}{2n} = \Theta(1+ \frac{\alpha}{2}) = \Theta(1+ \alpha)
		\end{align*}

		\begin{gather*}
			\alpha \text{ costante} \\
			n = O(m) \\
			n \leq k \cdot m \\
			\alpha = \frac{n}{m} \leq k \\
			\Rightarrow \Theta(1 + \alpha) = \Theta(1)
		\end{gather*}
	\end{itemize}
	\item $h \colon U \to \{ 0,1,\dots,m-1 \} \Rightarrow h(x) = 0$
\end{itemize}

\subsubsection{Funzioni Hash} Una \emph{funzione hash} deve soddisfare la proprietà di
\emph{hashing uniforme}, ossia 
\begin{quote}
	``Ogni chiave ha la stesso probabilità $\frac{1}{m}$ di essere mandata in una
	qualsiasi delle $m$ celle, indipendentemente dalle chiavi inserite precedentemente.''
\end{quote} 
Consideriamo:
\begin{itemize}
	\item $k \in [0,1)$ ($0 \leq k < 1$), $k$ chiave, estratta in modo indipendente dalla distribuzione
	uniforme (\underline{non realistica}).
	\item Allora $h(k) = \floor{mk}$ soddisfa la proprietà di \emph{hashing uniforme}.
\end{itemize} 

L'ipotesi di hash uniforme semplice dipende dalle probabilità con cui vengono estratti gli elementi da
inserire; probabilità che in genere non sono note.
Le funzioni hash che descriveremo assumono che le chiavi siano degli interi non negativi.

\paragraph{Metodo della divisione} 
\begin{gather*}
	U = \{ 0, 1, \dots, \abs{\cup} - 1 \} \\
	h(k) \in \{0, 1, \dots, m - 1 \} \\
	h(k) = k \bmod m
\end{gather*}

\begin{itemize}
	\item $m = 2^p$ caso pessimo;
	\item $m = 2^p - 1$ caso non buono. $2^p$ cifre base.
\end{itemize}
La soluzione migliore è quella di scegliere chiavi lontane dalle potenze di 2, meglio
ancora se numeri primi.

\paragraph{Metodo della moltiplicazione}
$$k \in U$$
$$0 < A < 1 \text{ fissato}$$
$$h(k) = m(k A \bmod 1) \qquad \text{Miglior } A : \frac{\sqrt{5}-1}{2}$$
\begin{align*}
	m & = 2^p \quad w = \# \text{bit parola} \\
	A & = \frac{q}{2^w} \quad 0 < q < 2 \\
	& m(k A \bmod 1) \\
	& = m \left( k \frac{q}{2^w} \bmod 1 \right) && (\text{\emph{shift} di $w$ bit, prendo la parte decimale} \\
	& && k a \bmod 1 \text{ e la moltiplico per } m = 2^p) 
\end{align*}

\subsubsection{Hashing Universale} 
Per avere una distribuzione più uniforme delle chiavi nelle liste e non dipendente dall'input,
possiamo usare la \emph{randomizzazione}. \par
Insieme $H$ di funzioni di hash. Scelgo randomicamente $h \in H$. \par
Sotto certe ipotesi ottengo per \texttt{Search}:
$$\Theta(1 + \alpha)$$

\paragraph{Def (Hashing universale)} $\forall k_1,k_2 \in U, \ k_1 \neq k_2$
\begin{gather*}
	\left| \left\{ h \in H : h(k_1) = h(k_2) \right\} \right| \leq \frac{|H|}{m} \\
	prob(h(k_1) = h(k_2)) = \frac{\left| \left\{ h \in H : h(k_1) = h(k_2) \right\} \right|}{|H|} \leq \frac{1}{m}
\end{gather*}

\paragraph{Teorema} Con il \emph{chaining}, $H$ è universale $ \forall k \in U, \ j = h(k)$
\[ \text{Costo medio } \Theta(1+\alpha)
	\begin{cases}
		k \text{ non è in } T \rightarrow E[n_j] \leq \alpha \\
		k \text{ è in } T \rightarrow E[n_j] \leq 1 + \alpha
	\end{cases}
\]