\subsection{Open Addressing} \label{hash:openaddessing}

$h(k,i)$: $k$ è la chiave, $i$ è il tentativo.\par
Provo con $h(k,0)$: se capito in una cella occupata, provo con $h(k,1)$, poi $h(k,2)$
e così via, fino a che non trovo una cella libera.

Per esplorare tutta la tabella:
$$h(k,0), h(k,1), \dots, h(k,m-1) \quad \forall k \in U$$
che è una permutazione di 
$$\{0,1,\dots,m-1\}$$

\begin{codebox}
\Procname{$\proc{Insert}(T,x)$}
\li $i \gets 0$
\li \Repeat
\li     $j \gets h(\attrib{x}{key},i)$
\li     \If $(T[j] = \const{nil})$ \Or $(T[j] = \const{deleted})$ 
        	\Comment posizione libera
\li     \Then
            $T[j] \gets x$
\li         \Return $j$
        \End
\li     $i \gets i + 1$
\li \Until $i = m$
\li \Error
\end{codebox}
\begin{codebox}
\Procname{$\proc{Search}(x,k)$}
\li \If $(x = \const{nil})$ \Or $(\attrib{x}{key} = k)$
\li \Then
		\Return $x$
\li \ElseIf $k < \attrib{x}{key}$
\li \Then	
		\Return \proc{Search}$(\attrib{x}{left},k)$
\li \ElseNoIf
\li 	\Return \proc{Search}$(\attrib{x}{right},k)$
	\End
\end{codebox}
\begin{codebox}
\Procname{\proc{Delete}$(T,x)$}
\li $T[\attrib{x}{\id{key}}] \gets \const{nil}$
	\Comment $\Theta(1)$
\end{codebox}
L'\emph{Open Addressing} risulta una soluzione inefficiente in caso
avvengano molte cancellazioni.

\subsubsection{Hashing Uniforme}

Per ogni elemento di input, tutte ($m!$) le sequenze di ispezione
sono equiprobabili.

\subsubsection{Funzioni di Hash}
\begin{enumerate}
    \item \textbf{Ispezione lineare}. Sia $h'(k)$ funzione di hash ``ordinaria''. Se 
        ricado in una cella occupata, mi sposto su quella immediatamente successiva.
        $$h(k,i) = (h'(k)+i) \bmod m$$
        Caratteristiche:
        \begin{itemize}
            \item è semplice;
            \item poche permutazioni ($m$ dipende solo da $h'(k)$);
            \item causa addensamenti di celle occupate (\emph{addensamento primario}).
        \end{itemize}
    \item \textbf{Ispezione quadratica}. Fisso $h'(k)$.
    $$h(k,i) = (h'(k) + c_1i + c_2i^2) \bmod m, \qquad c_2 > 0$$

    \item \textbf{Doppio Hash}. Fisso $h_1(k)$, $h_2(k)$
    $$h(k,i) = (h_1(k) + i \cdot h_2(k)) \bmod m$$
    \emph{Osservazioni}:
    \begin{itemize}
        \item I salti sono di dimensione $h_2(k)$ all'incrementare di $i$; 
        \item Ci sono $m^2$ sequenze di ispezione;
        \item $h_2(k)$ e $m$ primi tra loro $ \quad (MCD = 1)$;
        \item $h(k,0),h(k,1),\dots,h(k,m-1)$ è permutazione di $\{0,1,\dots,m-1\}$;
        \item $i, i'<m \quad h(k,i) = h(k,i') \Rightarrow i = i' \ \quad (\text{\emph{iniettività}})$
        $$h(k,i) \colon \{0,\dots,m-1\} \to \{0,\dots,m-1\}$$
        $$\text{\emph{iniettiva}} \Rightarrow \text{\emph{biiettiva}}$$
        \begin{gather*}
            h(k,i) = h(k,i') \\
            (h_1(k) + ih_2(k)) \bmod m = (h_1(k) + i'h_2(k)) \bmod m \\
            ((i - i')h_2(k)) \bmod m = (i h_2(k) - i'h_2(k)) \bmod m = 0 \\
            (i - i') \bmod m = 0 \\
            i \geq i' \quad i - i' < m - 1 \\ 
            \Rightarrow i - i' = 0 \\
            \Rightarrow i = i'
        \end{gather*}
        Scelgo $m = 2^p, \ h_2(k) = 1 + 2h'(k)$, $h'(k)$ qualunque.\par
        \emph{es.} $h_2(k) = 1 + h'(k \bmod m') \quad$ con $m' < m, m$ primo
    \end{itemize}
    \end{enumerate}

\paragraph{Costo}
Il costo della \texttt{Search} con \emph{hashing uniforme} si può riassumere come segue.
$$0 \leq \alpha = \frac{n}{m} \leq 1$$

\paragraph{Ricerca di una chiave non presente}
\begin{enumerate}[label=(\alph*)]
    \item $\frac{1}{1-\alpha} \quad$ se $\alpha < 1$
    \item $m \quad$ se $\alpha = 1$ 
\end{enumerate}

\subparagraph{Probabilità di ispezionare la i-esima cella}
\begin{center}
    \begin{tabular}{c|l}
        \textbf{cella} & \textbf{probabilità} \\
        \hline
        $i = 0$ & 1 \\
        $i = 1$ & prob. cella 0 occupata: $\frac{n}{m}$ \\
        $i = 2$ & prob. cella 1 occupata: $\frac{n}{m} \cdot \frac{n-1}{m-1}$ \\
        $\dots$ \\
        $i$ & $\frac{n}{m} \cdot \frac{n-1}{m-1} \dots \frac{n-i+1}{m-i+1} \leq \alpha \cdot \alpha \dots \alpha = \alpha^i$
    \end{tabular}
\end{center}

Valore atteso per \verb|#| tentativi
$$1 + \alpha + \alpha^2 + \dots + \alpha^{i-1} + \dots + \alpha^{m-1}$$

\begin{enumerate}[label=(\alph*)]
    \item $\alpha < 1 \Rightarrow \frac{1 - \alpha^m}{1 - \alpha} \leq \frac{1}{1-\alpha}$ 
    \item $m$ 
\end{enumerate}

\paragraph{Ricerca di una chiave presente}
\begin{enumerate}[label=(\alph*)]
    \item $\frac{1}{\alpha} \log \left( \frac{1}{1-\alpha} \right) \quad \alpha < 1$
    \item $1 + \log m \quad \alpha = 1$
\end{enumerate}

Costo atteso per $k_i$:
\begin{align*}
	& \frac{1}{1-\alpha_{i-1}} = \frac{1}{1-\frac{i-1}{m}} && \left(\alpha_{i-1} = \frac{i-1}{m} < 1\right) \\
	& = \frac{m}{m-i+1}
\end{align*}

Il costo atteso è la media per $i = 1,\dots,n$
\begin{align*}
    & = \frac{1}{n} \displaystyle\sum_{i=1}^{n} \frac{m}{m-i+1}
		= \frac{m}{n}\displaystyle\sum_{i=1}^{n} \frac{1}{m-i+1} && \left(\frac{m}{n} = \frac{1}{\alpha}\right)\\
    & = \frac{1}{\alpha} \displaystyle\sum_{j=m-n+1}^{m}\frac{1}{j}
\end{align*}

\begin{itemize}
    \item Se $\alpha < 1$ %SISTEMARE, <= di cosa?
    \begin{align*}
        & \leq \frac{1}{\alpha} \int_{n-m}^{m}\frac{1}{x} \mathrm{d}x \\ 
        & = \frac{1}{\alpha}\left( \log m - \log (m-n) \right) = \frac{1}{\alpha}\left( \log \frac{m}{m-n} \right) \\
        & = \frac{1}{\alpha} \log \frac{1}{\frac{m-n}{m}} \\ 
        & = \frac{1}{\alpha} \log \left( \frac{1}{1-\left( \frac{n}{m} \right)} \right)
            = \frac{1}{\alpha} \log \left( \frac{1}{1-\alpha } \right)
    \end{align*}

    \item Se $\alpha = 1$
    \begin{align*}
        \displaystyle\sum_{l = 1}^{m} \frac{1}{l} & = 1 + \displaystyle\sum_{l=2}^m \frac{1}{l} 
            \leq \int_1^m \frac{1}{x} \mathrm{d}x \\
        & = 1 + \left( \log m - \log 1 \right) = 1 + \log m
    \end{align*}
\end{itemize}

Confrontiamo le complessità dei due casi.

\begin{center}
    \begin{tabular}{l|l|l}
        $\alpha$ & $\frac{l}{1-\alpha}$ & $\frac{1}{\alpha} \log \left( \frac{1}{1-\alpha} \right)$ \\
        \hline
        $\alpha = 0.3$  & 1.43 & 1.19 \\
        $\alpha = 0.5$  & 2.00 & 1.39 \\
        $\alpha = 0.7$  & 3.33 & 1.72 \\
        $\alpha = 0.9$  & 10 & 2.56 \\
        $\alpha = 0.99$ & 100 & 4.65         
    \end{tabular}
\end{center}