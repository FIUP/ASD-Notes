\section{Lezione 14}

\subsection{Max Heap con coda dinamica}
Implementiamo un \texttt{Max Heap}, solo che è implementato con una
coda dinamica invece che con un normale array.

Ogni \emph{nodo} $x$ ha 3 campi dato:
\begin{itemize}[noitemsep]
    \item \texttt{x.left};
    \item \texttt{x.right};
    \item \texttt{x.p}.
\end{itemize}

$H$ è lo \emph{heap} ($\const{nil}$ se vuoto):
\begin{itemize}[noitemsep]
    \item \texttt{H.root};
    \item \texttt{H.size}.
\end{itemize}

Abbiamo \texttt{MaxHeapify} e \texttt{MaxHeapifyUp} invariate.

\begin{codebox}
\Procname{\proc{Insert}$(H, \id{node})$}
\li \If $\attrib{H}{size} = 0$
\li     \Then
            $\attrib{\id{node}}{p} \gets \const{nil}$
\li         $\attrib{H}{root} \gets \id{node}$
\li         $\attrib{H}{size} = 1$
\li     \Else
\li         $x \gets root$
\li         $\attrib{H}{size} \gets \attrib{H}{size} + 1$
\li         $p \gets \proc{bitvector}(\attrib{H}{size})$
            \Comment $p$ letto come vettore di bit
\li         $k \gets \# \text{bit di p più significativo a 1}$
\li         \For $i \gets k-1$ \kw{down} \To $1$
\li             \Do 
                    \If $p[i] = 0$
\li                     \Then
                            $x \gets \attrib{x}{left}$
\li                     \Else 
\li                         $x \gets \attrib{x}{right}$
                    \End
            \End
\li         \If $p[0] = 0$
\li             \Then
                    $\attrib{x}{left} \gets \id{node}$
\li             \Else
\li                 $\attrib{x}{right} \gets \id{node}$
            \End
    \End
\li $\attrib{\id{node}}{left} \gets \attrib{\id{node}}{right} \gets \const{nil}$
\li $\proc{MaxHeapifyUp}(H, \id{node})$
\end{codebox} 

Raccolta esercizi della lezione: \ref{exs:6-4-2018}.