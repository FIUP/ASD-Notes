\section{Programmazione Dinamica}
\subsection{Critica al D\&C}
\begin{center}
	\begin{tikzpicture}
	\draw (0,0) node (a) [anchor=north]{}
	-- (4,0) node (b) [anchor=north]{}
	-- (2,4) node (c) [anchor=south]{}
	-- cycle;
	\coordinate (r1) at ($(b)+(1,0)$);
	\coordinate (r2) at ($(c)+(1,0)$);
	\coordinate (l1) at ($(a)-(1,0)$);
	\coordinate (l2) at ($(c)-(1,0)$);
	\draw[-{>[length=3mm,width=2mm]},semithick] (r1) -- (r2) node [align=left,midway,right=.5cm,fill=white]{(C) bottom-up\\elaborazione soluzioni sottoistanze};
	\draw[{<[length=3mm,width=2mm]}-,semithick] (l1) -- (l2) node [align=right,midway,left=.5cm,fill=white]{(D) top-down\\generazione istanze};
	\node at (2,0) [align=center,below=.2cm,fill=white]{albero delle chiamate ricorsive};
	\end{tikzpicture}
\end{center}
Il processo di soluzione non ha memoria, quindi le soluzioni di sottoistanze vanno ricalcolate.

\paragraph{Esempio}
Vediamo uno "spreco usando D\&C: la sequenza di Fibonacci.
\begin{flalign*}
	& F(n) =
	\begin{cases}
	1 & \text{se } n = 0,1 \\
	F(n-1) + F(n-2) & \text{se } n \geq 2
	\end{cases} &
\end{flalign*}

\begin{codebox}
\Procname{\proc{Rec-Fib}$(n)$}
\li \If $(n = 0) \kw{ or }(n = 1)$
\li	\Then
		\Return 1
	\End
\li	\Return $\proc{Rec-Fib}(n-1) + \proc{Rec-Fib}(n-2)$
\end{codebox}

Ad esempio, con $n = 5$
\begin{center}
	\begin{tikzpicture}[tree]
	\Tree
	[.5
		[.4
			[.3
				[.2
					[.1 ]
					[.0 ]
				]
				[.1 ]
			]
			[.2
				[.1 ]
				[.0 ]
			]
		]
		[.3
			[.2
				[.1 ]
				[.0 ]
			]
			[.1 ]
		]
	]
	\end{tikzpicture}
\end{center}
Vengono ricalcolate $F(3)$ e $F(2)$.

\paragraph{Complessità}
\begin{flalign*}
	T(n) =
	\begin{cases}
	0 & \text{se } n = 0,1 \qquad \text{(il ''return'' costa 0)} \\
	T(n-1) + T(n-2) + 1 & \text{se } n \geq 2 \qquad \quad \text{(il ''+'' costa 1)}
	\end{cases}
\end{flalign*}
\begin{align*}
	T(n) & \geq T(n-1) + T(n-2) + 1 \\
	& \geq 2T(n-2) + 1 \\
	& \geq 2(2T(n-2-2)+1) + 1 \\
	& = 2^2 T(n-2-2) + 2 + 1 \\
	& \geq 2^i T(n-2 \cdot i) + \sum_{j=0}^{i-1} 2^j
\end{align*}
\begin{align*}
	i_0 \rightarrow i = \left \lfloor{\frac{n}{2}} \right \rfloor & \\
	\text{se n è pari: } & \quad 2^\frac{n}{2} T(n-2 \frac{n}{2}) = 2^\frac{n}{2} T(0) \\
	\text{se n è dispari: } & \quad \left \lfloor{\frac{n}{2}} \right \rfloor = \frac{n-1}{2} \\
	& \quad 2^\frac{n-1}{2} T(n-2 \frac{n-1}{2}) =  2^\frac{n-1}{2} T(1)
\end{align*}
Otteniamo
$$T(n) \geq \sum_{j=0}^{\left \lfloor{\frac{n}{2}} \right \rfloor -1} 2^j = \Theta(2^\frac{n}{2})$$
In verità,
$$T(n) = \Theta \left(\left(\frac{1+\sqrt{5}}{2}\right)^n \right)$$
\bigskip
\par Vediamo ora una versione iterativa:
\begin{codebox}
\Procname{$\proc{It-Fib}(n)$}
\li \If $(n = 0)$ \Or $(n = 1)$
\li	\Then
		\Return $1$
	\End
\li	$F[0] \gets 1$
\li	$F[1] \gets 1$ 
\li \For $i \gets 2$ \To $n$
\li \Do
		$F[i] \gets F[i-1] + F[i-2]$
	\End
\li \Return $F[n]$
\end{codebox}

\paragraph{Complessità} $\Theta(n)$
\bigskip
\par La programmazione dinamica salta la fase top-down.

\subsection{Memoizzazione}
È un ibrido tra il D\&C e la programmazione dinamica che vuole mantenere la fase top-down pur cercando di ricordare le soluzioni ai sottoproblemi

\paragraph{Def} Un algoritmo \emph{memoizzato} è costituito da due subroutine distinte:
\begin{enumerate}[label={\arabic*)}]
	\item \emph{routine di inizializzazione}: risolve direttamente i casi base e inizializza una struttura dati che contiene le soluzioni ai casi base e gli elementi per tutte le sottoistanze da calcolare, inizializzate ad un valore di default
	\item \emph{routine ricorsiva}: esegue il codice D\&C preceduto da un test sulla struttura dati per verificare se la soluzione è già stata calcolata e memorizzata. Se sì, si ritorna, altrimenti la si calcola ricorsivamente e la si memorizza nella struttura.
\end{enumerate}

\paragraph{Esempio} Riprendiamo l'esempio di prima sulla sequenza di Fibonacci.

\begin{codebox}
\Procname{\proc{Init-Fib}$(n)$}
\li \If $(n = 0) \kw{ or }(n = 1)$
\li	\Then
		\Return 1
	\End
\li	$F[0] \gets 1$
\li	$F[1] \gets 1$ 
\li \For $i \gets 2 \To n$
\li \Do
		$F[i] \gets 0$
	\End
\li \Return $\proc{Rec-Fib}(n)$
\end{codebox}
\subparagraph{Complessità} $\Theta(n)$

\begin{codebox}
\Procname{\proc{Rec-Fib}$(i)$}
\li \If $F[i] = 0$
\li	\Then
		$F[i] \gets \proc{Rec-Fib}(i-2) + \proc{Rec-Fib}(i-1)$
	\End
\li	\Return $F[i]$
\end{codebox}

Ad esempio, con $n = 5$
\begin{center}
	\begin{tikzpicture}[tree]
	\Tree
	[.5
		[.3
			[.1 ]
			[.2
				[.0 ]
				[.1 ]
			]
		]
		[.4
			[.2 ]
			[.3 ]
		]
	]
	\end{tikzpicture}
\end{center}
Questa volta, $F(3)$ e $F(2)$ non vengono ricalcolate. Abbiamo $n$ foglie e $n-1$ nodi interni ($n$ parte da $0$).
