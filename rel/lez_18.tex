\subsection{Problemi di Ottimizzazione}
\begin{align*}
	& I = \text{insieme delle istanze} \\
	& S = \text{insieme delle soluzioni} \\
	& \Pi \subseteq I \times S \\
	& \forall i \in I, \ S(i) = \{s \in S : (i,s) \in \Pi\} = \text{insieme delle soluzioni ammissibili} \\
	& c \colon S \to \mathbb{R} = \text{funzione di costo} \\
	& \text{Determinare, data } i \in I, \ s^* \in S(i) : c(s^*) = \min (\text{/}\max)\{c(s) : s \in S(i)\}
\end{align*}

\paragraph{Problema della raggiungibilità su un grafo orientato}
\begin{align*}
	& I = \{\langle G=(V,E), \ u, \ v \rangle \ : \ V \subseteq \mathbb{N}, \ V \text{ finito}, \ E \subseteq V \times V, \ u,v \in V\} \\
	& S = \{\langle v_1,v_2,\dots,v_k \rangle \ : \ k \geq 1, \ v_i \in \mathbb{N} \quad \forall 1 \leq i \leq k\} \cup \{\varepsilon\} \qquad (\varepsilon = \text{cammino vuoto}) \\
	& \Big(i = \langle G=(V,E), \ u, \ v \rangle, \ s \Big) \in \Pi \Longleftrightarrow
	\begin{cases}
	S = \varepsilon, \exists \text{un cammino tra } u \text{ e } v \text{ in } G \\
	\begin{aligned}
		& S = \langle v_1,v_2,\dots,v_k \rangle, \ v_1=u, \ v_k=v, \\
		& \qquad (v_i,v_{i+1}) \in E \quad \forall 1 \leq i \leq k
	\end{aligned}
	\end{cases} \\
	& c(\langle v_1,v_2,\dots,v_k \rangle) = k-1 \\
	& c(\varepsilon) = +\infty
\end{align*}

\paragraph{Caratteristiche}
Un problema di ottimizzazione, per essere risolto con la programmazione dinamica, deve avere le seguenti caratteristiche:
\begin{itemize}[noitemsep]
	\item struttura ricorsiva;
	\item esistenza di sottoistanze ripetute;
	\item spazio di sottoproblemi ''piccolo''.
\end{itemize}

\paragraph{Paradigma Generale}
\begin{enumerate}
	\item Caratterizza la struttura di una soluzione ottima $s^*$ in funzione di soluzione ottime $s_1^*,s_2^*,\dots,s_k^*$ di sottoistanze di taglia inferiore.
	\item Determina una relazione di ricorrenza del tipo $c(s^*) = f(c(s_1^*),\dots,c(s_k^*))$.
	\item Calcola $c(s^*)$ impostando il calcolo in maniera bottom-up (oppure memoizzando).
	\item Mantiene informazioni strutturali aggiuntive che permettono di ricostruire $s^*$.
\end{enumerate}

\subsection{Problemi su Stringhe}
\paragraph{Def}
Dato un alfabeto finito $\Sigma$, una \emph{stringa}
$$X = \langle x_1,x_2,\dots,x_m\rangle, \quad x_i \in \Sigma \quad \forall 1 \leq i \leq m$$
è una concetazione finita di simboli in $\Sigma$.
\begin{align*}
	& m = \abs X = \text{lunghezza di } X \\
	& \Sigma^* = \text{insieme di tutte le stringhe di lunghezza finita costruibili su } \Sigma \\
	& \varepsilon = \text{stringa vuota}
\end{align*}
Data una stringa $X$, il \emph{prefisso} di $X$ è
$$X_i = \langle x_1,x_2,\dots,x_i \rangle, \quad 1 \leq i \leq m$$
Data una stringa $X$, il \emph{suffisso} di $X$ è
$$X^i = \langle x_i,x_{i+1},\dots,x_m \rangle, \quad 1 \leq i \leq m$$
Per convenzione $X_0 = X^{m+1} = \varepsilon$

\paragraph{Def}
Data una stringa $X$, la \emph{sottostringa} di $X$ è
$$X_{i \dots j} = \langle x_i,x_{i+1},\dots,x_j \rangle, \quad 1 \leq i \leq j \leq m$$
Per convenzione $X_{i \dots j} = \varepsilon \quad$ se $i > j$
\bigskip

\noindent \verb|#| possibili sottostringhe di una stringa con $m$ caratteri:
\begin{gather*}
	\begin{array}{cccccl}
		\dbinom{m}{2} & + & m & + & 1 & = \dfrac{m(m+1)}{2} = \Theta(m^2) \\
		\uparrow & & \uparrow & & \uparrow & \\
		i \neq j & & i = j & & \varepsilon &
	\end{array}
\end{gather*}
Lo spazio delle sottostringhe ''non è troppo grande''.

\paragraph{Def}
Data una stringa
$$X = \langle x_1,x_2,\dots,x_m\rangle \in \Sigma^*$$
e
$$Z = \langle z_1,z_2,\dots,z_k\rangle \in \Sigma^*$$
si dice che $Z$ è \emph{sottosequenza} di $X$ se $\exists$ una successione crescente di indici
$$1 \leq i_1 \leq i_2 \leq \dots \leq i_k \leq m \ : \ z_j = x_{ij} \quad \forall 1 \leq j \leq k$$
\paragraph{Esempio}
\begin{align*}
	& X = \langle a,b,c,b,b,d \rangle \\
	& Z_1 = \langle a,b,c \rangle = X_{1 \dots 3} \\
	& Z_2 = \langle a,c,b \rangle \qquad i_1 = 1, \quad i_2 = 3, \quad i_3 = 4 \text{ o } 5 \\
	& Z_3 = \langle b,b \rangle = X_{4 \dots 5} \qquad i_1 = 2, \quad i_2 = 5
\end{align*}
\verb|#| possibili sottosequenze di una stringa con $m$ caratteri:
\begin{gather*}
	\begin{array}{lcl}
		\displaystyle\sum_{k=0}^{m} & \dbinom{m}{k} & = 2^m \\
		& \uparrow & \\
		& \text{\scriptsize stringhe lunghe } \scriptstyle k & \\[-5pt]
		& \text{\scriptsize prese da un insieme} & \\[-5pt]
		& \text{\scriptsize di } \scriptstyle m \ \text{\scriptsize elementi} &
	\end{array}
\end{gather*}



