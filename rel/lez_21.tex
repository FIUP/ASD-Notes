\subsection{Longest Increasing Subsequence (LIS)}
\paragraph{Def}
Dato un alfabeto $\Sigma$ totalmente ordinato $(\forall a,b \in \Sigma \quad a < b \text{ o } a = b \text{ o } a > b)$ e dato $X = \angleset{x_1,x_2,\dots,x_n}$, si dice che $Z = \angleset{z_1,z_2,\dots,z_k}$ è \emph{sottosequenza crescente} di $X$ $(Z = IS(X))$.

\paragraph{Problema di ottimizzazione}
Determinare la \emph{più lunga sottosequenza crescente} di $X$ $(Z = LIS(X))$

\paragraph{Esempio}
\begin{flalign*}
	& X = \angleset{5,2,4,3,7,8} & \\
	& Z = LIS(X) = \angleset{2,3,7,8} & \\
	& Z' = LIS(X) = \angleset{2,4,7,8} &
\end{flalign*}

\paragraph{Tentativo}
Data $X$, calcolo $LIS(X_i) \ \forall 0 \leq i \leq n$ \par
$Z = \angleset{z_1,z_2,\dots,z_k} = LIS(X_i)$
\begin{itemize}
	\item caso fortunato: $z_k < X_{i+1}$
	\item caso sfortunato: $z_k \geq X_{i+1}$
\end{itemize}

\paragraph{Def} $Z = \overline{LIS}(X_i)$ è la più lunga tra le $IS(X_i)$ con $Z = \angleset{z_1,z_2,\dots,z_k} = \angleset{x_{i_1},x_{i_2},\dots,x_{i_k}}$ con $i_k = i$.



\paragraph{Esempio}
\begin{flalign*}
	& X = \angleset{8,2,5,1,3} & \\
	& LIS(X_4) = \angleset{2,5} & \\
	& \overline{LIS}(X_4) = \angleset{1} &
\end{flalign*}
In generale, $LIS$ e $\overline{LIS}$ sono problemi molto diversi.

\paragraph{Osservazione}
$\abs{LIS(X)} = \displaystyle\max_{1 \leq i \leq n}\{\abs{\overline{LIS}(X_i)}\}$

\paragraph{Dimostrazione}
\begin{itemize}
	\item $\abs{LIS(X)} \leq \displaystyle\max_{1 \leq i \leq n}\{\abs{\overline{LIS}(X_i)}\}$
	\begin{flalign*}
		& Z = LIS(X) = \angleset{x_{i_1},x_{i_2},\dots,x_{i_k}} & \\
		& Z = \overline{LIS}(X_{i_k}) & \\
		& \Rightarrow \abs Z \leq \displaystyle\max_{1 \leq i \leq n}\{\abs{\overline{LIS}(X_i)}\} &
	\end{flalign*}
	\item $\abs{LIS(X)} \geq \displaystyle\max_{1 \leq i \leq n}\{\abs{\overline{LIS}(X_i)}\}$ \\
	Si dimostra analogamente al punto precedente.
\end{itemize}

\subsubsection{Proprietà di Sottostruttura Ottima}
\begin{enumerate}
	\item \label{lis:1} caso base: $\overline{LIS}(X_1) = \angleset{x_1} \quad (= LIS(X_1))$
	\item \label{lis:2} $i > 1$
	\begin{enumerate}
		\item \label{lis:2.a} $\forall j : 1 \leq j \leq i \quad x_j \geq x_i$
		\begin{flalign*}
			& \overline{LIS}(X_i) = \angleset{x_i} \quad (= LIS(X_i)) &
		\end{flalign*}
		\item \label{lis:2.b} $\exists \overline{j} : 1 \leq n \overline{j} \leq i, \quad  x_{\overline{j}} < x_i$
		\begin{flalign*}
			& \abs{\overline{LIS}(X_i)} \geq 2 & \\
			& \overline{LIS}(X_i) = \angleset{z_1,z_2,\dots,z_k} = \angleset{Z_{k-1},x_i} \text{ con } Z_{k-1} : \abs{Z_{k-1}} = \displaystyle\max_{1 \leq j < i}\{\overline{LIS}(X_j) : x_j < x_i\} &
		\end{flalign*}
	\end{enumerate}
\end{enumerate}

\paragraph{Dimostrazione}
\begin{enumerate}
	\item[\ref{lis:1}.] banale
	\item[\ref{lis:2}.] $i > 1$
	\begin{enumerate}
		\item[{\hyperref[lis:2.a]{(a)}}] $\forall j < i \quad x_j < x_i$
		\begin{flalign*}
			& \angleset{x_i} = IS(X_i) \text{ e chiaramente non può essere } \abs Z > 1 &
		\end{flalign*}
		Supponiamo per assurdo che
		\begin{flalign*}
			& \abs Z = \abs{\overline{LIS}(X_i)} > 1 & \\
			& Z = \angleset{z_1,z_2,\dots,z_k}, \quad k > 1 & \\
			& Z = \angleset{x_{i_1},x_{i_2},\dots,x_{i_k}} & \\
			& i_k = i & \\
			& \Rightarrow i_{k-1} < i_k = i &
		\end{flalign*}
		Assurdo perchè allora avrei
		\begin{flalign*}
			& x_{i_{k-1}} \geq x_{i_k} &
		\end{flalign*}
		che contraddice l'ipotesi che $Z$ è una sequenza crescente.
		\item[{\hyperref[lis:2.b]{(b)}}] Si dimostra analogamente al sottocaso precedente.
	\end{enumerate}
\end{enumerate}

\subsubsection{Ricorrenza sui Costi}
\begin{flalign*}
	& l(i) = \abs{\overline{LIS}(X_i)} & \\
	& l(i) =
	\begin{cases}
	1 & \text{se } i = 1 \\
	1 + \displaystyle\max_{1 \leq j < i}\{l(j) : x_j < x_i\} & \text{se } i > 1
	\end{cases} &
\end{flalign*}
Per costruire la sottosequenza
\begin{flalign*}
	& \overline{LIS}(X_i) = \begin{cases}
	\angleset{x_i} & (1) \\
	\angleset{\overline{LIS}(X_{\overline{j}}),x_i} \quad \text{ con } 1 \leq \overline{j} < i & (2)
	\end{cases} &
\end{flalign*}
l'informazione addizionale è:
\begin{itemize}
	\item $prev(i) = \begin{cases}
	0 & \text{nel caso (1)}  \\
	\overline{j} & \text{nel caso (2)}
	\end{cases}$
	\item $len = \displaystyle\max_{1 \leq i \leq n}\{l(i)\}$
	\item $end = i \quad \text{se } \overline{LIS}(X_i) = LIS(X)$ \\ (mantiene l'indice dell'ultimo carattere della LIS)
\end{itemize}

\paragraph{Pseudocodice bottom-up}
\begin{codebox}
\Procname{$\proc{LIS}(X)$}
\li $L[1] \gets 1$
\li $len \gets 1$
\li $end \gets 1$
\li $prev[1] = 0$
\li \For $i \gets 2 \To n$
\li \Do
		$L[i] \gets 1$
\li		$prev[i] \gets 0$
\li		\For $j \gets 1 \To i-1$
\li		\Do \If $x_j < x_i$
\li			\Then \If $L[i] < 1 + L[j]$
\li				\Then
					$L[i] \gets 1 + L[j]$
\li					$prev[i] \gets j$
				\End
			\End
		\End
\li		\If $len < L[i]$
\li		\Then
			$len = L[i]$
\li			$end = i$
		\End
	\End
\li \Return $(len, prev, end)$
\end{codebox}
\begin{codebox}
\Procname{$\proc{R-Print}(X,prev,i)$}
\li \If $prev[i] \neq 0$
\li \Then $\proc{R-Print}(X,prev,prev[i])$
	\End
\li $\proc{Print}(x_i)$
\end{codebox}
\paragraph{Complessità}
$\displaystyle\sum_{i=2}^{n}\sum_{j=1}^{i-1} 1 = \sum_{i=2}^{n} (i-1) = \frac{n(n-1)}{2} = \Theta(n^2)$

