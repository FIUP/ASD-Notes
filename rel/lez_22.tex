\subsection{Completamento a Palindromo (CP)}
\paragraph{Def} Una stringa $Z = \angleset{z_1,z_2,\dots,z_m}$ è \emph{palindroma} se $z_{1+h} = z_{m-h} \quad \forall 0 \leq h \leq m-1$.

\paragraph{Problema}
Data $X = \angleset{x_1,x_2,\dots,x_n}$, un \emph{complemento palindromo} di $X$ è una stringa $Z = CP(X) = \angleset{z_1,z_2,\dots,z_m}$ con $m \geq n$ tale che:
\begin{enumerate}[label={\arabic*)}]
	\item $Z$ è palindroma;
	\item $X$ è sottosequenza di $Z$
\end{enumerate}
(cioè, $Z$ è un palindromo che contiene $X$ come sottosequenza).

\paragraph{Esempio}
\begin{flalign*}
	& X = \angleset{a,c,b} & \\
	& Z = \angleset{a,c,b,b,c,a} & \\
	& Z' = \angleset{a,b,c,c,b,a} &
\end{flalign*}

\paragraph{Osservazione}
$\abs X = n \leq \abs Z \leq 2n = 2 \abs X$

\subsubsection{Problema di Ottimizzazione}
Determinare $Z = CP(X)$ di lunghezza minima. \\
Spazio dei sottoproblemi: $X_{i \dots j}$ (cioè lo spazio delle sottostringhe di $X$). \par
Determinare un algoritmo bottom-up quadratico nella lunghezza di $X$. \\
$\forall i,j : 1 \leq i \leq j \leq n$, determinare il minimo $Z = CP(X_{i \dots j})$.

\subsubsection{Proprietà di Sottostruttura Ottima}
Casi base:
\begin{enumerate}
	\item $i = j \quad$ (1 carattere)
	\begin{flalign*}
		& X = \angleset{x_i} & \\
		& CP(X_{i \dots j}) = X_{i \dots j} & \\
		& \abs{CP(X_{i \dots j})} = \abs{X_{i \dots j}} &
	\end{flalign*}
	\item $j = i + 1 \quad$ (2 caratteri)
	\begin{flalign*}
		& X_{i \dots j} = \angleset{x_i,x_{i+1}} &
	\end{flalign*}
	\begin{enumerate}
		\item $x_i = X_{i+1}$
		\begin{flalign*}
			& CP(X_{i \dots i+1}) = X_{i \dots i+1} & \\
			& \abs{CP(X_{i \dots i+1})} = 2 &
		\end{flalign*}
		\item $x_i \neq x_{i+1}$
		\begin{flalign*}
			& \text{Un possibile } CP(X_{i \dots i+1}) \text{ è } \angleset{x_i,x_{i+1},x_i} & \\
			& \abs{CP(X_{i \dots i+1})} = 3 &
		\end{flalign*}
	\end{enumerate}
\end{enumerate}
Caso generale:
\begin{enumerate}
		\item[] $X_{i \dots j} = \angleset{x_i,x_{i+1},\dots,x_j}$
		\begin{enumerate}
			\item $x_i = x_j$
			\begin{flalign*}
				& Z = CP(X_{i \dots j}) & \\
				& z_1 = z_k = x_i \ (= x_j) & \\
				& Z_{2 \dots k-1} = CP(X_{i+1 \dots j-1}) &
			\end{flalign*}
			\item $x_i \neq x_j$
			Ci sono due possibili soluzioni:
			\begin{enumerate}
				\item \label{cp:cg:b.1} $z_1 = z_k =x_i \text{ e } Z_{2 \dots k-1} = CP(X_{i+1 \dots j})$
				\item \label{cp:cg:b.2} $z_1 = z_k =x_j \text{ e } Z_{2 \dots k-1} = CP(X_{i \dots j-1})$
			\end{enumerate}
		\end{enumerate}
\end{enumerate}

\paragraph{Dimostrazione}
\begin{enumerate}
	\item[{\hyperref[cp:cg:b.1]{(b).i.}}] Suppongo per assurdo che
	\begin{flalign*}
		& z_1 = z_k \neq x_i & \\
		& \Rightarrow X_{i \dots j} \text{ è sottosequenza di} & \\
		& Z_{2 \dots k-1} \text{ che è palindroma e più corta di 2 rispetto a } Z &
	\end{flalign*}
	Assurdo perchè $Z$ è la più corta!
	\item[{\hyperref[cp:cg:b.1]{(b).ii.}}] Si dimostra analogamente al sottocaso precedente.
\end{enumerate}

\subsubsection{Ricorrenza sulle Lunghezze}
Definisco
\begin{flalign*}
	& l(i,j) = \abs{CP(X_{i \dots j})} & \\
	& l(i,j) = \begin{cases}
	1 & \text{se } i = j \\
	2 & \text{se } j = i + 1 \text{ e } x_i = x_j \\
	3 & \text{se } j = i + 1 \text{ e } x_i \neq x_j \\
	2 + l(i+1,j-1) & \text{se } j > i + 1 \text{ e } x_i = x_j \\
	2 + \min\{l(i+1,j),l(i,j-1)\} & \text{se } j > i + 1 \text{ e } x_i \neq x_j
	\end{cases}
\end{flalign*}
Per calcolare $l(i,j)$ mi possono servire tre valori:
\begin{gather*}
	L = 
	\setlength{\arraycolsep}{0pt}
	\left[\begin{array}{ccccc}
	\qquad & & & & \qquad \\
	& (i,j{-}1) & \leftarrow & (i,j) & \\
	& & \swarrow & \downarrow & \\
	& (i{+}1,j{-}1) & & (i{+}1,j) & \\
	& & & &
	\end{array}\right]
\end{gather*}
Riempio la tabella per diagionali, dall'alto verso il basso e da sinistra verso destra.
\begin{codebox}
\Procname{$\proc{SPC}(X)$}
\li $n \gets \attrib{X}{length}$
\li \For $i \gets 1$ \To $n-1$
\li \Do
		$L[i,j] \gets 1$
\li		\If $x_i = x_{i+1}$
\li		\Then
			$L[i,i+1] \gets 2$
\li		\Else
\li			$L[i,i+1] \gets 3$
		\End
	\End
\li	$L[n,n] \gets 1$
\li \For $l \gets 3$ \To $n$
		\Comment scansione diagonale con $l$ indice della diagionale
\li	\Do
		\For $i \gets 1$ \To $n-l+1$
\li		\Do
			$j \gets i+l-1$
\li			\If $x_i = x_j$
\li			\Then
				$L[i,j] \gets 2 + L[i+1,j-1]$
\li			\Else
\li				$L[i,j] \gets 2 + \min\{L[i+1,j],L[i,j-1]\}$
			\End
		\End
	\End
\li \Return $L[1,n]$
\end{codebox}
\paragraph{Complessità}
\begin{flalign*}
	& \displaystyle\sum_{i=1}^{n-1} 1 + \sum_{l=3}^{n}\sum_{i=1}^{n-l+1} 1 = \sum_{i=1}^{n-1} 1 + \sum_{l=3}^{n} (n-l+1) = \sum_{i=1}^{n-1} 1 + \sum_{i=2}^{n-1} (n-i) = & \\
	& = \sum_{i=1}^{n-1} (n-i) = \sum_{j=1}^{n-1} j = \frac{n(n-1)}{2} = \Theta(n^2) &
\end{flalign*}