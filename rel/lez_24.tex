\subsection{Compressione dei Dati: Codici di Huffman}
La compressione può essere di due tipi:
\begin{itemize}[noitemsep]
	\item lossy, cioè con perdita di informazione (e.g. immagine, video);
	\item lossless, cioè senza perdita di informazione (e.g. testo).
\end{itemize}

\paragraph{Esempio}
File di caratteri con frequenze associate:
\medskip

\begin{tabular}{c|cccccc}
	               & a  & b  & c  & d  & e & f \\ \hline
	frequenza (\%) & 45 & 13 & 12 & 16 & 9 & 5
\end{tabular}
\bigskip

La codifica ASCII usa 8 bit per carattere. \\
File con 100K caratteri $\rightarrow$ 800 Kbit
\bigskip

Definisco una codifica che usa 3 bit per carattere:
\medskip

\begin{tabular}{c|cccccc}
	         & a   & b   & c   & d   & e   & f \\ \hline
	codeword & 000 & 001 & 010 & 011 & 100 & 101
\end{tabular}
\begin{flalign*}
	& \begin{array}{ll}
	\text{File con 100K caratteri} \rightarrow & \text{300 Kbit (risparmio più del 50\%)} \\
	& + \text{ spazio per una tabella di conversione/decodifica}
	\end{array} &
\end{flalign*}

\begin{tabular}{c|c|}
	\cline{2-2}	
	0 & \textquotesingle a\textquotesingle \\ \cline{2-2}
	1 & \textquotesingle b\textquotesingle \\ \cline{2-2}
	2 & \textquotesingle c\textquotesingle \\ \cline{2-2}
	3 & \textquotesingle d\textquotesingle \\ \cline{2-2}
	4 & \textquotesingle e\textquotesingle \\ \cline{2-2}
	5 & \textquotesingle f\textquotesingle \\ \cline{2-2}	
\end{tabular}
\bigskip

Nella posizione $i$ si trova il carattere la cui codifica è $i$ stesso (in binario).

\paragraph{Def}
\begin{flalign*}
	& C = \text{alfabeto dei simboli presenti nel file} & \\
	& \text{funzione di encoding } e \colon C \to {0,1}^* &
\end{flalign*}
Proprietà che deve avere $e$:
\begin{itemize}
	\item invertibile $\rightarrow$ iniettiva: se $a,b \in C, a \neq b \Rightarrow e(a) \neq e(b)$;
	\item ammettere algoritmi efficienti: $e,e^{-1}$.
\end{itemize}
Una funzione di encoding che associa ad ogni carattere di $C$ una stringa di 0 e 1 della stessa lunghezza si chiama \emph{fixed length encoding}.
\bigskip

Finora ho ignorato la frequenza dei caratteri.
\paragraph{Idea} Associare a caratteri più frequenti codeword più corte e, di conseguenza, a caratteri meno frequenti codeword più lunghe.
\medskip

\begin{tabular}{c|cccccc}
				   & a  & b  & c  & d  & e   & f   \\ \hline
	frequenza (\%) & 45 & 13 & 12 & 16 & 9   & 5   \\
	codeword       & 0  & 00 & 01 & 1  & 100 & 101
\end{tabular}

\paragraph{Problema} $e(aa) = e(b)$ \\
Non sono come decodificare in modo univoco perchè esistono delle codeword che sono prefissi di altre codeword.
\bigskip

La codifica che sto cercando deve:
\begin{itemize}
	\item avere lunghezza variabile della codeword;
	\item affinchè si possa avere l'invertibilità, il codice deve essere \underline{libero da prefissi} (\emph{codice prefisso}\footnote{Attenzione: con il termine ''codice prefisso'' si intende un codice senza prefissi.}), cioè $\nexists a,b \in C : e(a) $ è un prefisso di $e(b)$.
\end{itemize}

\paragraph{Soluzione}\mbox{}\\
\begin{tabular}{c|cccccc}
				   & a  & b   & c   & d    & e    & f   \\ \hline
	frequenza (\%) & 45 & 13  & 12  & 16   & 9    & 5   \\
	codeword       & 0  & 101 & 100 & 111  & 1101 & 1100
\end{tabular}
\bigskip

Questo è un codice a lunghezza variabile libero da prefissi ed è ottimo tra tutti i codici che associano una codeword ad ogni singolo carattere.
\bigskip

Spazio occupato da questo encoding:
\begin{flalign*}
	& 100\text{K} \left(\frac{45}{100} \cdot 1 + \frac{13}{100} \cdot 3 + \frac{12}{100} \cdot 3 + \frac{16}{100} \cdot 3 + \frac{9}{100} \cdot 4 + \frac{5}{100} \cdot 4 \right) = 224 \text{ Kbit} &
\end{flalign*}
300 Kbit $\rightarrow$ risparmio $\approx$ 25\%
\medskip

Serve memorizzare dell'informazione aggiuntiva per la decodifica:
\begin{center}
	\begin{tikzpicture}[tree] 
	\Tree
	[.{}
		\edge node[midway,left]{0};
		[.a ]
		\edge node[midway,right]{1};
		[.{}
			\edge node[midway,left]{0};
			[.{}
				\edge node[midway,left]{0};
				[.c ]
				\edge node[midway,right]{1};
				[.b ]
			]
			\edge node[midway,right]{1};
			[.{}
				\edge node[midway,left]{0};
				[.{}
					\edge node[midway,left]{0};
					[.f ]
					\edge node[midway,right]{1};
					[.e ]	
				]
				\edge node[midway,right]{1};
				[.d ]
			]
		]
	]
	\end{tikzpicture}
\end{center}
Stringa codificata: 110001001101 \\
Stringa decodificata: face
\bigskip

Un qualsiasi codice binario può essere rappresentato in modo compatto con un albero binario. \par
Un codice prefisso è associato a un \emph{albero di codifica} T in cui i caratteri da codificare appaiono tutti alle foglie.
\medskip

$\forall c \in C$ ho $f(c)$ frequenza
\smallskip

Definisco
$$d_T(c) = \text{profondità di c in T}$$
\begin{flalign*}
	& d_T(\textnormal{\textquotesingle}\text{a}\textnormal{\textquotesingle}) = 1 & \\
	& d_T(\textnormal{\textquotesingle}\text{b}\textnormal{\textquotesingle}) = d_T(\textnormal{\textquotesingle}\text{c}\textnormal{\textquotesingle}) = d_T(\textnormal{\textquotesingle}\text{d}\textnormal{\textquotesingle}) = 3 & \\
	& d_T(\textnormal{\textquotesingle}\text{e}\textnormal{\textquotesingle}) = d_T(\textnormal{\textquotesingle}\text{f}\textnormal{\textquotesingle}) = 4
\end{flalign*}
La profondità corrisponde alla lunghezza della codeword associata.
\medskip

Funzione di costo:
$$B(T) = \displaystyle\sum_{c \in C} f(c) \cdot d_T(c)$$
La dimensione del file compresso è
$$\abs{F_c} = \frac{B(T) \cdot \abs F}{100}$$
dove
$$\abs F = \text{taglia del file compresso (in \# di caratteri)}$$
$B(T)$, a meno di una costante, è il fattore di compressione $\min B(T)$.
\paragraph{Proprietà}
Un albero ottimo è sempre \underline{pieno} (cioè i nodi interni hanno due figli).
\bigskip

Spazio di problemi: spazio di alberi pieni con $n$ foglie $(n = \abs C)$
\paragraph{Idea}
Prendo due nodi con frequenza minore, gli unisco in un nodo che avrà come frequenza la somma delle due, ripeto $n-1$ volte. \\
In questo modo, i nodi con la frequenza maggiore verranno aggiunti per ultimi alla cima dell'albero, quindi avranno una profondità bassa, ovvero la lunghezza dalla codeword corta, proprio come volevo.
\medskip

$Q$: coda di priorità (Heap) di nodi con chiave $f(z)$
\smallskip

Attributi di un nodo:
\begin{itemize}[nosep,noitemsep]
	\item $z.left$
	\item $z.right$
	\item $z.f$
\end{itemize}

\paragraph{Pseudocodice}
\begin{codebox}
\Procname{$\proc{Huffman}(C,f)$}
\li	$n \gets \abs C$
\li	$Q \gets \emptyset$
\li	\For \kw{each} $c \in C$
		\Comment inizializzazione
\li	\Do
		$z \gets \proc{New-Node}()$
			\Comment crea un nuovo nodo
\li		$\attrib{z}{f} \gets f(c)$
\li		$\attrib{z}{left} \gets \const{nil}$
\li		$\attrib{z}{right} \gets \const{nil}$
\li		$\proc{Insert}(Q,z)$
			\Comment $\Theta(\log n)$
	\End
\li	\For $i \gets 1$ \To $n-1$
\li	\Do
		$x \gets \proc{Extract-Min}(Q)$
\li		$y \gets \proc{Extract-Min}(Q)$
\li		$z \gets \proc{New-Node}()$
\li		$\attrib{z}{left} \gets x$
\li		$\attrib{z}{right} \gets y$
\li		$\attrib{z}{f} \gets \attrib{x}{f} + \attrib{y}{f}$
\li		$\proc{Insert}(Q,z)$
	\End
\end{codebox}
\paragraph{Complessità}
$\Theta\left(\displaystyle\sum_{i=1}^{n-1} \log (n-i)\right) = \Theta(n \log n)$

\paragraph{Esempio}
Riprendiamo l'esempio iniziale e applichiamo l'algoritmo.
\begin{center}
	\begin{tikzpicture}[tree]
	\Tree
	[.100
		\edge node[midway,left]{0};
		[.45/a ]
		\edge node[midway,right]{1};
		[.55
			\edge node[midway,left]{0};
			[.25
				\edge node[midway,left]{0};
				[.c/12 ]
				\edge node[midway,right]{1};
				[.b/13 ]
			]
			\edge node[midway,right]{1};
			[.30
				\edge node[midway,left]{0};
				[.14
					\edge node[midway,left]{0};
					[.f/5 ]
					\edge node[midway,right]{1};
					[.e/9 ]	
				]
			\edge node[midway,right]{1};
			[.d/16 ]
			]
		]
	]
	\end{tikzpicture}
\end{center}
\begin{tabular}{c|cccccc}
	         & a  & b   & c   & d    & e    & f   \\ \hline
	codeword & 0  & 101 & 100 & 111  & 1101 & 1100
\end{tabular}
\bigskip

\noindent $B(T) = \displaystyle\left(\frac{45}{100} \cdot 1 + \frac{13}{100} \cdot 3 + \frac{12}{100} \cdot 3 + \frac{16}{100} \cdot 3 + \frac{9}{100} \cdot 4 + \frac{5}{100} \cdot 4 \right) = 224$

\subsubsection{Proprietà di Scelta Greedy}
Sia $C$ un alfabeto e siano $x$ e $y$ i caratteri in $C$ di frequenza minore. \\
Allora esiste un codice prefisso ottimo $T$ in cui $x$ e $y$ sono foglie attaccate alo stesso padre (sorelle).

\paragraph{Dimostrazione}\mbox{}\\
Sia $T^*$ una soluzione ottima arbitraria $\Rightarrow B(T^*)$ è minima.
\smallskip

\noindent Non so dove siano $x$ e $y$ in $T^*$
\smallskip

\noindent Siano $b$ e $c$ le foglie sorelle di profondità massima in $T^*$
\begin{enumerate}
	\item \label{huffman:1} $d_{T^*}(b) = d_{T^*}(c) \geq d_{T^*}(x),d_{T^*}(y)$
	\item $f(x) \leq f(b), \quad f(y) \leq f(c)$
	\begin{center}
		\begin{tikzpicture}
		[edge from parent/.style={draw,edge from parent path={(\tikzparentnode) -- (\tikzchildnode)}}]
		\node (a) {$T^*$}
		child {node (b) [draw,circle]{x}}
		child {edge from parent[draw=none] node{\vdots}
			child {edge from parent[draw=none] node{\vdots}
				child {edge from parent[draw=none] node[draw,circle]{}
					child {node (d) [draw,circle]{b}}
					child {node (e) [draw,circle]{c}}
				}
			}
		}
		child {node (c) [draw,circle]{y}};
		\node (a1) [right=4cm of a]{$T'$}
		child {node (b1) [draw,circle]{b}}
		child {edge from parent[draw=none] node{\vdots}
			child {edge from parent[draw=none] node{\vdots}
				child {edge from parent[draw=none] node[draw,circle]{}
					child {node (d1) [draw,circle]{x}}
					child {node (e1) [draw,circle]{c}}
				}
			}
		}
		child {node (c1) [draw,circle]{y}};
		\node (a2) [right=4cm of a1]{$T''$}
		child {node (b2) [draw,circle]{b}}
		child {edge from parent[draw=none] node{\vdots}
			child {edge from parent[draw=none] node{\vdots}
				child {edge from parent[draw=none] node[draw,circle]{}
					child {node (d2) [draw,circle]{x}}
					child {node (e2) [draw,circle]{y}}
				}
			}
		}
		child {node (c2) [draw,circle]{c}};
		\draw[->] ($(a) + (1.5,0)$) -- ($(a1) - (1.5,0)$) node[midway,above]{\scriptsize scambio $x$ e $b$};
		\draw[->] ($(a1) + (1.5,0)$) -- ($(a2) - (1.5,0)$) node[midway,above]{\scriptsize scambio $y$ e $c$};
		\end{tikzpicture}
	\end{center}
	\begin{flalign*}
		& T^* \rightarrow T' & \\
		& B(T^*) \geq B(T') & \\
		& B(T^*) - B(T') \geq 0 & \\
	\end{flalign*}
	\begin{align*}
		B(T^*) - B(T') & = \sum_{c \in C} f(c)\, d_{T^*}(c) - \sum_{c \in C} f(c)\, d_{T'}(c)  \\
		& = f(b)\, d_{T^*}(b) + f(x)\, d_{T^*}(x) - f(b)\, \underbrace{d_{T'}(b)}_{d_{T^*}(b)} - f(x)\, \underbrace{d_{T'}(x)}_{d_{T^*}(x)} \\
		& = f(b)\, d_{T^*}(b) + f(x)\, d_{T^*}(x) - f(b)\, d_{T^*}(b) - f(x)\, d_{T^*}(x) \\
		& = f(b) \big(d_{T^*}(b) - d_{T^*}(x)\big) - f(x) \big(d_{T^*}(b) - d_{T^*}(x)\big) \\
		& = \big(\underbrace{f(b) - f(x)}_{\geq 0 \text{ perchè } f(x) \leq f(b)}\big) \big (\underbrace{d_{T^*}(b) - d_{T^*}(x)}_{\geq 0 \text{ per il punto }  \ref{huffman:1}}\big) \geq 0
	\end{align*}
	$T' \rightarrow T'' \quad $ si dimostra analogamente.
\end{enumerate}

\subsubsection{Proprietà di Sottostruttura Ottima}
Sia $T$ un codice prefisso ottimo che contiene la scelta greedy (sui caratteri $x$ e $y$). \\
Sia $z$ un nuovo carattere (cioè $\notin C$) con $f(z) = f(x) + f(y)$. \\
Allora il codice prefisso $T' = T \setminus \{x,y\}$ è ottimo per $C \setminus \{x,y\} \cup \{z\}$

\begin{center}
	\begin{tikzpicture}
	[edge from parent/.style={draw,edge from parent path={(\tikzparentnode) -- (\tikzchildnode)}}]
	\node (a) [draw,circle]{}
	child { node (b) [draw,circle]{$x$}}
	child { node (c) [draw,circle]{$y$}};
	\node [below=1mm of b]{$f(x)$};
	\node [below=1mm of c]{$f(y)$};
	\coordinate (d) at ($ (a) + (-1.5,0)$);
	\coordinate (e) at ($ (a) + (0,2.3)$);
	\coordinate (f) at ($ (a) + (1.5,0)$);
	\draw (a) -- (d) -- (e) -- (f) -- (a);
	\node [above=1mm of e]{$T$};
	\node (a1) [right=6cm of a,draw,circle]{$z$};
	\node [below=1mm of a1]{$f(z)$};
	\coordinate (d1) at ($ (a1) + (-1.5,0)$);
	\coordinate (e1) at ($ (a1) + (0,2.3)$);
	\coordinate (f1) at ($ (a1) + (1.5,0)$);
	\draw (a1) -- (d1) -- (e1) -- (f1) -- (a1);
	\node [above=1mm of e1]{$T'$};
	\draw[->] ($(a) + (2,1)$) -- ($(a1) + (-2,1)$);
	\end{tikzpicture}
\end{center}
$T$ ha $n$ foglie. \\
$T'$ ha $n-1$ foglie.
\smallskip

\noindent Cerco una relazione tra  $B(T)$ e $B(T')$.

\paragraph{Osservazione}
\begin{enumerate}
	\item $\forall c \in C \setminus \{x,y\}, \ c \neq z : d_{T'}(c) = d_T(c)$
	\item $d_{T'}(z) = d_T(x) - 1$
\end{enumerate}

\paragraph{Dimostrazione}
\begin{align*}
	B(T) & = \sum_{c \in C} f(c)\, d_T(c) & \big(d_T(x) = d_T(y)\big)\\
	& = \sum_{c \in C \setminus \{x,y\}} f(c)\, d_T(c) + \big(f(x) + f(y)\big) d_T(x) \\
	& = \sum_{c \in C \setminus \{x,y\}} f(c)\, d_T(c) + \big(\underbrace{f(x) + f(y)}_{f(z)}\big) \big(\underbrace{d_T(x)-1}_{d_{T'}(z)}\big) + \big(f(x)+f(y)\big) \\
	& = \sum_{c \in C \setminus \{x,y\} \cup \{z\}} f(c)\, d_{T'}(c) + f(x) + f(y) \\
	& = B(T') + f(x) + f(y)
\end{align*}
Suppongo per assurdo che $T$ non sia il codice prefisso ottimo per il sottoproblema $C \setminus \{x,y\} \cup \{z\}$. \\
Allora $\exists T''$ di costo strettamente inferiore.
\begin{flalign*}
	& B(T) = \underbrace{B(T'')}_{< B(T')} + f(x) + f(y) &
\end{flalign*}
Assurdo!

